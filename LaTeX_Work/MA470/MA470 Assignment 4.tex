\documentclass{article}[12pt,a4paper]
\usepackage{a4wide}
%\usepackage{fullpage}
\usepackage{tikz}
\usepackage{amssymb}
\usepackage{bbm}

\title{MA470 Assignment 4}
\author{Richard Douglas}
\date{February 27,  2014} %defaults to \today

\begin{document}
  \maketitle
  \noindent The final page of this submission consists of an appendix with derivations of general reusable results that 
  are used throughout my solutions.\newline
  \begin{enumerate}
  
  %Question 1
  \item[\textbf{Exercise 2.7}]
  (a) At time T, the payoff of the European contract is
  $$\Lambda(S(T)) = \left\{ 
  			\begin{array}{cl}
  			S(T) - K_1 & : S(T) \le K_1, \\
  			0               & : K_1 < S(T) < K_2, \\
  			S(T) - K_2 & : S(T) \ge K_2.
  			\end{array} 
  \right.$$
  In the first and third scenarios, the holder receives a stock worth $S(T)$ but must give up 
  the respective fixed strike amount. In the second scenario, the holder is forced to buy a stock for $S(T)$
  but can immediately sell it back to the market for that amount
  since the model assumes that a seller can always find a buyer.
  
  Clearly this contract can be replicated by a portfolio consisting of a short European put with strike $K_1$ and
  a long European call with strike $K_2$.
  
  The value of the contract at time $t$ with spot $S$ is thus
  $$V(t, S) = C(t, S, K_2; r, q) - P(t, S, K_1; r, q)$$
  $$= e^{-q\tau}S\cdot(N(d_+^*(S/K_2,\tau)) + N(-d_+^*(S/K_1,\tau))) $$
  $$      - e^{-r\tau}\cdot(K_2\cdot N(d_-^*(S/K_2,\tau) + K_1\cdot N(-d_-^*(S/K_1,\tau)))$$
  
  (b) In general, to have a present value of $0$, $K_1$ and $K_2$ must satisfy the relation
  $$Se^{(r - q)\tau} = \frac
  {K_2\cdot N(d_-^*(S/K_2,\tau) + K_1\cdot N(-d_-^*(S/K_1,\tau))}
  {N(d_+^*(S/K_2,\tau)) + N(-d_+^*(S/K_1,\tau))}$$
  
  In particular, if $K_1 = K_2$, then we have an ordinary forward contract with fair strike
  $Se^{(r-q)\tau}$. 
  \pagebreak
  
  
  
  %Question 2
  \item[\textbf{Exercise 2.21}]
  (a) 
  $$V^{cp}(t,S) = e^{-r(T_1 - t)}\widetilde{E}_{t, S}\left[ (P_{T_1} - K_1)^{+} \right]$$
  where $P_{T_1}$ is a strictly decreasing continuous function of random variable $S(T_1)$. 
  $$\lim_{S(T_1) \to 0}{S(T_2)} = \lim_{S(T_1) \to 0}{S(T_1)e^{\mu(T_2 - T_1) + \sigma(W(T_2) - W(T_1))}} = 0$$
  $$\therefore \lim_{S(T_1) \to 0}{P_{T_1}(S(T_1))} = K_2e^{-r(T_2 - T_1)}$$
  Similarly,
  $$\lim_{S(T_1) \to \infty}{S(T_2)} = \infty$$
  $$\therefore \lim_{S(T_1) \to \infty}{P_{T_1}(S(T_1))} = 0$$
  Thus $P_{T_1}(S(T_1)) \in (0,  K_2e^{-r(T_2 - T_1)})$ so the call on a put is worthless
  if $K_1 \ge K_2e^{-r(T_2 - T_1)}.$ \newline
  
  Assuming that $K_1 \in  (0,  K_2e^{-r(T_2 - T_1)})$, there exists a unique value $S_1^*$ 
  such that 
  $$P_{T_1}(S_1^*) = K_1$$
  Thus,
  $$V^{cp}(t,S) = e^{-r(T_1 - t)}\widetilde{E}_{t, S}\left[ (P_{T_1} - K_1) \mathbbm{1}_{\{ S(T_1) < S_1^* \}} \right]$$
  $$= e^{-r(T_1 - t)}( \widetilde{E}_{t, S}\left[ P_{T_1} \mathbbm{1}_{\{ S(T_1) < S_1^* \}}\right]
      -  K_1 \cdot \widetilde{P}_{t, S}(  S(T_1) < S_1^* ))$$
  Applying the Tower Property,
  $$=  e^{-r(T_1 - t)}(\widetilde{E}_{t, S}\left[ e^{-r(T_2 - T_1)} \widetilde{E}_{T_1, S(T_1)} 
  \left[ (K_2 - S(T_2)) \mathbbm{1}_{\{ S(T_2) < K_2 \}} \mathbbm{1}_{\{ S(T_1) < S_1^* \}} \right] \right]
  -  K_1 \cdot \widetilde{P}_{t, S}(  S(T_1) < S_1^* ))$$
  $$\mathbf{=  e^{-r(T_2 - t)}\widetilde{E}_{t, S}\left[ 
   (K_2 - S(T_2)) \mathbbm{1}_{\{ S(T_1) < S_1^*, S(T_2) < K_2 \}} \right]
  -  e^{-r(T_1 - t)} K_1 \cdot \widetilde{P}_{t, S}(  S(T_1) < S_1^* )}$$ \newline
  By \textbf{Result 6},
  $$e^{-r(T_2 - t)}K_2\widetilde{E}_{t, S}\left[  \mathbbm{1}_{\{ S(T_1) < S_1^*, S(T_2) < K_2 \}} \right]
  = e^{-r(T_2 - t)}K_2\widetilde{P}_{t, S}(S(T_1) < S_1^*, S(T_2) < K_2 )$$
  $$= e^{-r(T_2 - t)}K_2 \cdot N_2(-d_-^*(S/S_1^*, T_1 - t), -d_-^*(S/K_2, T_2 - t); \sqrt{\frac{T_1 - t}{T_2 - t}})$$
  
  By \textbf{Result 8},
  $$e^{-r(T_2 - t)}\widetilde{E}_{t, S}\left[ S(T_2) \mathbbm{1}_{\{ S(T_1) < S_1^*, S(T_2) < K_2 \}} \right]$$
  $$=e^{-r(T_2 - t)}Se^{(r - q)(T_2 - t)} N_2(-d_+^*(S/S_1^*, T_1 - t), -d_+^*(S/K_2,T_2 - t), \sqrt{\frac{T_1 - t}{T_2 - t}})$$
  $$= Se^{-q(T_2 - t)} N_2(-d_+^*(S/S_1^*, T_1 - t), -d_+^*(S/K_2,T_2 - t), \sqrt{\frac{T_1 - t}{T_2 - t}})$$
  By \textbf{Result 1},
  $$e^{-r(T_1 - t)} K_1 \cdot \widetilde{P}_{t, S}(  S(T_1) < S_1^* ) = e^{-r(T_1 - t)} K_1 \cdot (1 - N(d_-(S/S_1^*, T_1 - t)))$$
  $$=  e^{-r(T_1 - t)} K_1 \cdot N(-d_-(S/S_1^*, T_1 - t))$$
  
  \pagebreak
  
  \noindent Define
  $$a_\pm := d_\pm^*(S/S_1^*, T_1 - t)$$
  $$b_\pm := d_\pm^*(S/K_2, T_2 - t)$$
  Then
  $$V^{cp}(t,S) = e^{-r(T_2 - t)}K_2 \cdot N_2(-a_-, -b_-; \sqrt{\frac{T_1 - t}{T_2 - t}})$$ 
  $$- Se^{-q(T_2 - t)} N_2(-a_+, -b_+, \sqrt{\frac{T_1 - t}{T_2 - t}})$$ 
  $$- e^{-r(T_1 - t)} K_1 \cdot N(-a_-)$$
  \hfill (assuming that $K_1 \in  (0,  K_2e^{-r(T_2 - T_1)})$ and is $0$ otherwise.)
  
  (b) For $t < T_1$,
  $$\Delta(t, S) = \frac{\partial V^{cp}}{\partial S}$$
  $$ = - N_2(-a_+, -b_+, \sqrt{\frac{T_1 - t}{T_2 - t}})$$ 
  $$+ (e^{-r(T_2 - t)}K_2 \frac{\partial}{\partial S}N_2(-a_-, -b_-; \sqrt{\frac{T_1 - t}{T_2 - t}})
  - Se^{-q(T_2 - t)} \frac{\partial}{\partial S} N_2(-a_+, -b_+, \sqrt{\frac{T_1 - t}{T_2 - t}}))$$
  $$- \frac{\partial}{\partial S} e^{-r(T_1 - t)} K_1 \cdot N(-a_-)$$
  
  Applying \textbf{Result 10}, we have
  
  $$\Delta(t, S) = e^{-r(T_1 - t)} K_1 \cdot \frac{n(-a_-)}{S \sigma \sqrt{T_1 - t}} - 
  N_2(-a_+, -b_+, \sqrt{\frac{T_1 - t}{T_2 - t}})$$
   $$+ (e^{-r(T_2 - t)}K_2 \frac{\partial}{\partial S}N_2(-a_-, -b_-; \sqrt{\frac{T_1 - t}{T_2 - t}})
  - Se^{-q(T_2 - t)} \frac{\partial}{\partial S} N_2(-a_+, -b_+, \sqrt{\frac{T_1 - t}{T_2 - t}}))$$
  \hfill (assuming that $K_1 \in  (0,  K_2e^{-r(T_2 - T_1)})$ and is $0$ otherwise.)
  
  \pagebreak
  
  
  %Question 3
   \item[\textbf{Exercise 4.2}]
  The righthandside is equal to
  $$C(S; K, r, q) = \left\{
  			\begin{array}{ll}
  			\frac{K}{\lambda_+ - 1}(\frac{\lambda_+ - 1}{\lambda_+})^{\lambda_+}
  			(\frac{S}{K})^{\lambda_+} = \frac{S^*}{\lambda_+}(\frac{S}{S^*})^{\lambda_+}
  			& : 0 < S \le S^*,\\
  			S - K & : S^* < S.
  			\end{array}
  \right.$$
  \hfill (The location of equality can be changed because the price function is continuous at the point $S = S^*$.)
  
  where
  $$S^* = \frac{K\lambda_+}{\lambda_+ - 1}$$
  and
  $$\lambda_{+} = \frac{-(r - q - \sigma^2/2) + \sqrt{(r - q - \sigma^2/2)^2 + 2\sigma^2r}}{\sigma^2}$$
  
  whereas the lefthandside is equal to
  $$P(K; S, q, r) = \left\{
  			\begin{array}{ll}
  			\frac{S}{1 - \lambda_-'}(\frac{\lambda_-' - 1}{\lambda_-'})^{\lambda_-'}(\frac{K}{S})^{\lambda_-'} 
  			= -\frac{K^*}{\lambda_-'}(\frac{K}{K^*})^{\lambda_-'}& : K^* \le K, \\
  			S - K & : 0 < K < K^*
  			\end{array}
  \right.$$
  where 
  $$K^* = \frac{S\lambda_-'}{\lambda_-' - 1}$$
  and
  $$\lambda_{-}' = \frac{-(q - r - \sigma^2/2) - \sqrt{(q - r -\sigma^2/2)^2 + 2 \sigma^2q}}{\sigma^2}$$
  
  \textbf{Lemma 1:} $\lambda_+ + \lambda_-' = 1$
  
  Proof:
  $$\lambda_+ + \lambda_-' = 1 + \frac{\sqrt{(r - q - \sigma^2/2)^2 + 2\sigma^2r } 
  - \sqrt{(q - r - \sigma^2/2)^2 + 2\sigma^2q}}{\sigma^2}$$
  Note that
  $$((r - q - \sigma^2/2)^2 + 2\sigma^2r) - ((q - r - \sigma^2/2)^2 + 2\sigma^2q)$$
  $$= (r - q)^2 - (r - q)\sigma^2 + \sigma^4/4 + 2\sigma^2r - (q - r)^2 + (q - r)\sigma^2 - \sigma^4/4 - 2\sigma^2q$$
  $$= -(r - q)\sigma^2 + 2\sigma^2r + (q - r)\sigma^2 - 2\sigma^2q = 0$$
  Thus 
  $$\lambda_+ + \lambda_-' = 1 + 0 = 1$$
  
  \textbf{Lemma 2:} $K < K^* \iff S > S^*$
  
  Proof: 
  $$K < K^* \iff K < \frac{S\lambda_-'}{\lambda_-' - 1} \iff \frac{S}{K} > \frac{\lambda_-' - 1}{\lambda_-'}$$
  \hfill(Applying \textbf{Lemma 1}.)
  $$\iff  \frac{S}{K} > \frac{-\lambda_+}{1 - \lambda_+} \iff  S > \frac{K\lambda_+}{\lambda_+ - 1}
  \iff S > S^*$$
  \hfill(Note that it is assumed that $S > 0$ and $K > 0$.)
  
  (Lemma 2 also implies that $K \ge K^* \iff S \le S^*$)
  
  \pagebreak
  After applying \textbf{Lemma 2}, the only thing left to show is that
  $$\frac{S^*}{\lambda_+}(\frac{S}{S^*})^{\lambda_+} = -\frac{K^*}{\lambda_-'}(\frac{K}{K^*})^{\lambda_-'}$$
  This follows since
  $$\frac{S^*}{\lambda_+}(\frac{S}{S^*})^{\lambda_+} =
  \frac{K}{\lambda_+ - 1}(\frac{S(\lambda_+ - 1)}{K\lambda_+})^{\lambda_+} =
   \frac{K}{-\lambda_-'}(\frac{S(-\lambda_-')}{K(1 - \lambda_-')})^{1 - \lambda_-'}$$
   \hfill(by \textbf{Lemma 1}.)
   $$=  -\frac{K}{\lambda_-'}(\frac{S(\lambda_-')}{K(\lambda_-' - 1)})^{1 - \lambda_-'}
   = -\frac{K}{\lambda_-'}(\frac{K^*}{K})^{1 - \lambda_-'}
   $$ 
    $$=  -\frac{K}{\lambda_-'}(\frac{K^*}{K})(\frac{K}{K^*})^{\lambda_-'}
    = -\frac{K^*}{\lambda_-'}(\frac{K}{K^*})^{\lambda_-'}$$
    
    Thus we can conclude that
    $$P(K; S, q, r) = C(S; K, r, q)$$
  \pagebreak
  
  
  %Question 4
   \item[\textbf{Exercise 4.6}]
  We have the payoff function
  $$\Lambda(S) =  (K_1 - S)^+ + (S - K_2)^+$$
  Moreover we know that the value of the option satisfies the Black-Scholes PDE
  $$\mathcal{L}V + \frac{\partial V}{\partial t} = 0 \mbox{ (assuming that (t, S) is in the continuation domain,)}$$
  but $V = V(S)$ does not depend on $t$ so $\frac{\partial V}{\partial t} = 0$ and we have the ODE
  $$\frac{1}{2}\sigma^2S^2\frac{d^2 V}{dS^2} + (r - q)S\frac{dV}{dS} - rV = 0$$
  There are two scenarios in which the option should be exercised right away:
  \begin{enumerate}
  \item[i)] S is such that $K_1 - S$ is sufficiently large, or
  \item[ii)] S is such that $S - K_2$ is sufficiently large.  
  \end{enumerate}
  Thus there exist $S_1^*$ and $S_2^*$ such that the stopping domain is
  $$\mathcal{D} = \{ S \in (0, \infty) : S \le S_1^* \mbox{ or } S \ge S_2^* \}$$
  We thus have the ODE
  $$\frac{1}{2}\sigma^2S^2\frac{d^2 V}{dS^2} + (r - q)S\frac{dV}{dS} - rV = 0 \mbox{ for all } S \in \mathcal{D}^c$$
  with boundary conditions
  $$V(S_1^*) = \Lambda(S_1^*) = K_1 - S_1^*, \mbox{ and}$$ 
  $$V(S_2^*) = \Lambda(S_2^*) = S_2^* - K_2.$$
  The general solution to this ODE is
  $$V(S) = a_+S^{\lambda_+} + a_-S^{\lambda_-}$$
  where
  $$\lambda_\pm = \frac{-(r - q - \sigma^2/2) + \sqrt{(r - q - \sigma^2/2) + 2\sigma^2r}}{\sigma^2}.$$
  We can find $a_+$ and $a_-$ by solving the linear system:
  $$V(S_1^*) =  a_+S_1^{*\lambda_+} + a_-S_1^{*\lambda_-} =  K_1 - S_1^*$$
  $$V(S_2^*) =  a_+S_2^{*\lambda_+} + a_-S_2^{*\lambda_-} = S_2^* - K_2$$
  The above linear system has solution
  $$a_+ = \frac{S_2^{*\lambda_-}(K_1 - S_1^*) - S_1^{*\lambda_-}(S_2^* - K_2)}
  		{S_1^{*\lambda_+}S_2^{*\lambda_-} - S_1^{*\lambda_-}S_2^{*\lambda_+}}$$
  $$a_- = \frac{S_1^{*\lambda_+}(S_2^* - K_2) - S_2^{*\lambda_+}(K_1 - S_1^*)}
  		{S_1^{*\lambda_+}S_2^{*\lambda_-} - S_1^{*\lambda_-}S_2^{*\lambda_+}}$$
  \pagebreak
  
  To find $S_1^*$ and $S_2^*$ we can apply the smooth pasting condition. \newline
  
  First we need to find 
  $$V'(S) = a_+\lambda_+S^{(\lambda_+ - 1)} + a_-\lambda_-S^{(\lambda_- - 1)}$$
  
  Near $S = S_1^*$, $\Lambda(S) = K_1 - S$ and so
  $$\Lambda'(S_1^*) = -1$$
  Similarly,
  $$\Lambda'(S_2^*) = 1.$$
  By the smooth pasting condition, $S_1^*$ and $S_2^*$ satisfy the nonlinear system:
  $$V'(S_1^*) = a_+\lambda_+S_1^{*(\lambda_+ - 1)} + a_-\lambda_-S_1^{*(\lambda_- - 1)} = -1$$
   $$V'(S_2^*) = a_+\lambda_+S_2^{*(\lambda_+ - 1)} + a_-\lambda_-S_2^{*(\lambda_- - 1)} = 1$$
   It is unclear if we can proceed analytically at this point.
  \pagebreak
  
  
  \end{enumerate}
  \pagebreak
  \section*{Appendix}
  \textbf{Generalized $d$ function:}
  $$d_{\pm}^{*}(m,\tau) := \frac{ln(m) + (r - q \pm \sigma^2/2)\tau}{\sigma\sqrt\tau}$$
  \newline \textbf{European Call Without Dividend:} 
  $$C(t, S) = S\cdot N(d_+(S/K,\tau)) - e^{-r\tau}K\cdot N(d_-(S/K, \tau))$$
  \newline \textbf{European Call With Continuous Dividend:} 
  $$C(t, S ; r, q) = e^{-q\tau}S\cdot N(d^*_+(S/K, \tau)) - e^{-r\tau}K\cdot N(d^*_-(S/K,\tau))$$
  \newline \textbf{European Put-Call Parity}
  $$C(t,S;r,q) - P(t,S;r,q) = e^{-q\tau}S - e^{-r\tau}K$$
  \newline \textbf{European Put Without Dividend:}
  $$P(t, S) = C(t, S) + e^{-r\tau}K - S = e^{-r\tau}K\cdot N(-d_-(S/K,\tau)) - S\cdot N(-d_+(S/K,\tau))$$
  \newline \textbf{European Put With Dividend:}
  $$P(t, S; r, q) = e^{-r\tau}K\cdot N(-d^*_-(S/K,\tau)) - 
  	e^{-q\tau}S\cdot N(-d^*_+(S/K,\tau))$$
  \newline\textbf{Result 1:}
  $$\widetilde{P}_{t, S}(S(t) > K) = \widetilde{P}(\frac{S(T)}{S(t)} > \frac{K}{S}) = 
  \widetilde{P}(\widetilde{Z} > \frac{-ln(S/K) - (r - q - \sigma^2/2)(T - t)}{\sigma\sqrt{T - t}}) = N(d^*_-(S/K,T - t))$$ \newline
  \textbf{Result 2:}
  $$\widetilde{P}(S(t) > K) = \widetilde{P}(\frac{S(t)}{S(0)} > \frac{K}{S(0)}) = N(d_-^*(S(0)/K,t))$$ \newline
  \textbf{Result 3:}
  $$\widetilde{E}_{t\mbox{,}S}\left[ S^{\alpha}(T) \right] = 
   S(t)^\alpha e^{\alpha(r - q - \sigma^2/2)\tau} \widetilde{E}\left[ e^{\alpha \sigma \sqrt\tau \widetilde{Z}}\right] =
    S(t)^\alpha e^{\alpha(r - q - \sigma^2(1  - \alpha)/2 )\tau}$$ 
    \newline \textbf{Result 4:}
    $$
    E \left[ e^{aZ} \mathbbm{1}_{\{ Z > b \}} \right] = 
    \int_b^{\infty} { e^{az} \frac{1}{\sqrt{2\pi}} e^{-z^2/2} dz} = 
    e^{a^2/2}\int_{b}^{\infty}{\frac{1}{\sqrt{2\pi}}  e^{-(z - a)^2/2} dz} = 
    e^{a^2/2} \cdot (1 - N(b - a)) = e^{a^2/2} \cdot N(a - b)$$
    \textbf{Result 5:}
    $$\widetilde{P}_{t, S}(S(T_1) > K_1, S(T_2) > K_2) = N_2(d_-^*(S/K_1, T_1 - t), 
    		d_-^*(S/K_2, T_2 - t);\sqrt{\frac{T_1 - t}{T_2 - t}})$$
    \hfill (derived in MA470 notes, equation 2.88) \newline
    \textbf{Result 6:}
     $$\widetilde{P}_{t, S}(S(T_1) < K_1, S(T_2) < K_2) =
     \widetilde{P}_{t, S}(\widetilde{Z}_1 < -d_-(S/K_1, T_1 - t), \widetilde{Z}_2 < -d_-(S/K_2, T_2 - t))$$
     $$ = N_2(-d_-^*(S/K_1, T_1 - t), -d_-^*(S/K_2, T_2 - t);\sqrt{\frac{T_1 - t}{T_2 - t}})$$
     \pagebreak
     
     \noindent \textbf{Result 7:}
     $$\widetilde{E}_{t, S}\left[ S(T_2) \mathbbm{1}_{\{ S(T_1) > K_1, S(T_2) > K_2\}} \right]
     = Se^{(r - q)(T_2 - t)}N_2(d_+^*(S/K_1, T_1 - t), d_+^*(S/K_2,T_2 - t), \sqrt{\frac{T_1 - t}{T_2 - t}})$$
     \hfill (derived in MA470 notes, equation 2.89) \newline
     \textbf{Result 8:}
      $$\widetilde{E}_{t, S}\left[ S(T_2) \mathbbm{1}_{\{ S(T_1) < K_1, S(T_2) < K_2 \}} \right]
      = \widetilde{E}_{t, S}\left[ S(T_2) (1 -  \mathbbm{1}_{\{ S(T_1) > K_1\}})(1 -  \mathbbm{1}_{\{S(T_2) > K_2 \}}) 		
      \right]$$
      $$= \widetilde{E}_{t, S}\left[ S(T_2) \right] - 
      \widetilde{E}_{t, S}\left[ S(T_2) \mathbbm{1}_{\{ S(T_1) > K_1\}} \right] - 
      \widetilde{E}_{t, S}\left[ S(T_2) \mathbbm{1}_{\{S(T_2) > K_2 \}} \right] + 
      \widetilde{E}_{t, S}\left[ S(T_2) \mathbbm{1}_{\{ S(T_1) > K_1, S(T_2) > K_2\}} \right]$$
      $$= Se^{(r - q)(T_2 - t)} -  \widetilde{E}_{t, S}\left[S(T_1) \mathbbm{1}_{\{ S(T_1) > K_1\}} \right]
       \widetilde{E}\left[ \frac{S(T_2)}{S(T_1)} \right]
       - Se^{(r-q)(T_2 - t)}N(d_+^*(S/K_2, T_2 - t))$$ 
      $$+ Se^{(r - q)(T_2 - t)}N_2(d_+^*(S/K_1, T_1 - t), d_+^*(S/K_2,T_2 - t), \sqrt{\frac{T_1 - t}{T_2 - t}})$$
      \hfill ($\widetilde{E}_{t, S}\left[S(T_1) \mathbbm{1}_{\{ S(T_1) > K_1\}} \right] = 
      Se^{(r - q)(T_1 - t)}N(d_+^*(S/K_1, T_1 - t))
       ,\widetilde{E}\left[ \frac{S(T_2)}{S(T_1)} \right] = e^{(r - q)(T_2 - T_1)}$)
      $$= \mathbf{Se^{(r - q)(T_2 - t)}(1 - N(d_+^*(S/K_1, T_1 - t))- N(d_+^*(S/K_2, T_2 - t)) }$$
      $$\mathbf{ + N_2(d_+^*(S/K_1, T_1 - t), d_+^*(S/K_2,T_2 - t), \sqrt{\frac{T_1 - t}{T_2 - t}}))}$$ 
      (The idea is to now use
      $$P(X \le x, Y \le y) = 1 - P(X > x) - P(Y > y) + P(X > x, Y > y)$$
      with $-\widetilde{Z}_1$ and $-\widetilde{Z}_2$.) \newline
      
      \noindent In short, Result 8 states that 
      $$\widetilde{E}_{t, S}\left[ S(T_2) \mathbbm{1}_{\{ S(T_1) < K_1, S(T_2) < K_2 \}} \right]
      =Se^{(r - q)(T_2 - t)} N_2(-d_+^*(S/K_1, T_1 - t), -d_+^*(S/K_2,T_2 - t), \sqrt{\frac{T_1 - t}{T_2 - t}})$$ \newline
      \textbf{Result 9:}
      $$e^{-r\tau}K\frac{\partial}{\partial S} N(-d_-^*(S/K, \tau)) - 
      Se^{-q\tau}\frac{\partial}{\partial S} N(-d_+^*(S/K, \tau)) = 0$$
      \hfill (this follows from the Delta of a European Put option.) \newline
      \textbf{Result 10:}
      $$\frac{\partial}{\partial S}N(-d_\pm^*(S/K, \tau)) = \frac{\partial}{\partial S}
      N(-(\frac{ln(S/K) + (r - q \pm \sigma^2/2)\tau}{\sigma\sqrt\tau}))$$
     $$= \frac{- n(d_\pm^*(S/K, \tau))}{S \sigma \sqrt{\tau}}$$
     \hfill (where $n$ is the pdf of a standard Normal distribution.)
      
    
    
%    \textbf{Result 5:}
%    For $T_2 > T_1 > t$,
%    $$\widetilde P(\frac{S(T_1)}{S(t)} > \frac{K_1}{S}, \frac{S(T_2)}{S(t)} > \frac{K_2}{S})
%    = \widetilde P(\widetilde Z_1 > -d_-^*(S/K_1, T_1 - t), \widetilde Z_2 > -d_-^*(S/K_2, T_2 - t))$$
%    $$= \widetilde P(-\widetilde Z_1 < d_-^*(S/K_1, T_1 - t), -\widetilde Z_2 < d_-^*(S/K_2, T_2 - t))$$
%    Where $\widetilde Z_1 = \frac{W(T_1) - W(t)}{\sqrt{T_1 - t}}$, $\widetilde Z_2 = \frac{W(T_2) - W(t)}{\sqrt{T_2 - t}}$,
%    and $Cov(-\widetilde Z_1 ,-\widetilde Z_2) = Cov(\widetilde Z_1 ,\widetilde Z_2) = \frac{\sqrt{T_1 - t}}{\sqrt{T_2 - t}}$.
%    \pagebreak
%    \newline Thus Result 5 states that
%    $$\widetilde P(\frac{S(T_1)}{S(t)} > \frac{K_1}{S}, \frac{S(T_2)}{S(t)} > \frac{K_2}{S}) 
%    = N_2( d_-^*(S/K_1, T_1 - t), d_-^*(S/K_2, T_2 - t))$$
%    \hfill where $\rho = \frac{Cov(-\widetilde Z_1 ,-\widetilde Z_2)}{\sqrt{Var(-\widetilde Z_1) \cdot Var(-\widetilde Z_2)}} =  
%    \frac{\sqrt{T_1 - t}}{\sqrt{T_2 - t}}$ \newline
%    \newline \textbf{Result 6:}
%    For $T_2 > T_1 > t$,
%    $$\widetilde P(\frac{S(T_1)}{S(t)} < \frac{K_1}{S}, \frac{S(T_2)}{S(t)} > \frac{K_2}{S})
%    = \widetilde P(\frac{S(T_2)}{S(t)} > \frac{K_2}{S}) -  
%    \widetilde P(\frac{S(T_1)}{S(t)} > \frac{K_1}{S}, \frac{S(T_2)}{S(t)} > \frac{K_2}{S})$$
%    $$= N(d_-^*(S/K_2, T_2 - t)) - N_2( d_-^*(S/K_1, T_1 - t), d_-^*(S/K_2, T_2 - t) ; \frac{\sqrt{T_1 - t}}{\sqrt{T_2 - t}})$$
%     \newline \textbf{Result 7:}
%     $$\widetilde E_{t, S} \left[ S(T_2) \mathbbm{1}_{\{ S(T_1) > K_1, S(T_2) > K_2 \}} \right]
%     = Se^{(r - q)(T_2 - t)} \cdot N_2(d_+^*(S/K_1, T_1 - t), d_+^*(S/K_2, T_2 - t) ; \frac{\sqrt{T_1 - t}}{\sqrt{T_2 - t}})$$
%     \hfill (derived at equation $2.89$ of the MA470 notes.)
%     \newline \textbf{Result 8:}
%     $$\widetilde E_{t, S} \left[ S(T_2) \mathbbm{1}_{\{ S(T_1) < K_1, S(T_2) > K_2 \}} \right] =
%     \widetilde E_{t, S} \left[ S(T_2) \mathbbm{1}_{\{ S(T_2) > K_2 \}} \right] - 
%     \widetilde E_{t, S} \left[ S(T_2) \mathbbm{1}_{\{ S(T_1) > K_1, S(T_2) > K_2 \}} \right]$$
%     $$= Se^{(r - q)(T_2 - t)}\cdot (N(d_+^*(S/K_2, T_2 - t)) - 
%     N_2(d_+^*(S/K_1, T_1 - t), d_+^*(S/K_2, T_2 - t) ; \frac{\sqrt{T_1 - t}}{\sqrt{T_2 - t}}))$$
\end{document}