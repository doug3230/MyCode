\documentclass{article}[12pt,a4paper]
%\usepackage{a4wide}
\usepackage{fullpage}
\usepackage{tikz}

\title{MA372 Submission Problems}
\author{Richard Douglas}
\date{March 21,  2014}

\begin{document}
  \maketitle
  \begin{enumerate}
	  \item \textbf{Decision Variables}:
	  \begin{itemize}
	  \item $\mathbf{x_1}$: the units of toy 1 produced before Christmas
	  \item $\mathbf{x_2}$: the units of toy 2 produced before Christmas 
	  \item $\mathbf{y_1}$: a binary variable corresponding to the decision to produce toy 1 
	  \item $\mathbf{y_2}$: a binary variable corresponding to the decision to produce toy 2
	  \item $\mathbf{y_3}$: a binary variable corresponding to the decision to use factory 1 (factory 2 is used if $y_3$ is 0)
	  \end{itemize}
	  
	  \textbf{Objective Function}: \newline
	  We wish to maximize the profit generated (in \$) by the toys produced before Christmas. 
	  $$z = 10x_1 + 15x_2 - 50000y_1 - 80000y_2$$
	  
	   \textbf{Contraints}: \newline
	   $$x_1 \le My_1$$
	   $$x_2 \le My_2$$
	   $$y_i \in \{ 0, 1 \} \mbox{ for } i = 1, 2, 3$$
	   $$50x_1 + 40x_2 \le 500 + M(1 - y_3)$$
	   $$40x_1 + 25x_2 \le 700 + My_3$$
	   $$x_1, x_2 \ge 0$$
	   \hfill (where $M$ is a large positive constant)
	   
	   \textbf{Comment:} \newline
	   It may also make sense to impose $x_1$ and $x_2$ being integers as a constraint however if we solve the linear program, 
	   then the optimum solution will  have them in the thousands (in order to deal with the startup costs) so rounding to the 
	   nearest integer would give us the solution to the corresponding integer program. 
	   \pagebreak
	   
	   \item \section*{LP Relaxation}
	   \textbf{Optimum Tableau:}
	   \begin{center}
 	   \begin{tabular}{c | c | c c c c c | c}
            		Basis & $z$ & $x_1$ & $x_2$ & $x_3$ & $x_4$ & $x_5$ & RHS \\ \hline
  			           & $1$ & $0$ & $0$   & $0$ & $1/5$     & $6/5$               & $73/5$  \\ \hline
  			$x_3$ & $0$ & $0$ & $0$   & $1$    & $9/5$  & $-1/5$         & $17/5$ \\
  			$x_1$ & $0$ & $1$ & $0$  &  $0$    & $1/5$  & $1/5$         & $13/5$ \\
  			$x_2$ & $0$ & $0$ & $1$ &   $0$    & $-4/5$ & $1/5$         & $8/5$ \\
  	  \end{tabular}
  	  \end{center}
  	  \textbf{Optimal Objective Value:} $z = 73/5$
  	  
  	  \noindent \textbf{Optimum Solution:} $x_1 = 13/5$, $x_2 = 8/5$
  	  
  	  \noindent We can now branch with $x_2 \le 1$ (Subproblem 1) and $x_2 \ge 2$ (Subproblem 2).
  	  
  	  \textbf{Current Best $z$ Value:} $-\infty$
  	  
  	  \section*{Subproblem 1}
  	  \textbf{Post-Optimum Tableau in Standard Form:}
  	  \begin{center}
 	   \begin{tabular}{c | c | c c c c c c | c}
            		Basis & $z$ & $x_1$ & $x_2$ & $x_3$ & $x_4$ & $x_5$ & $x_6$ & RHS \\ \hline
  			           & $1$ & $0$ & $0$   & $0$ & $1/5$     & $6/5$ & $0$ & $73/5$  \\ \hline
  			$x_3$ & $0$ & $0$ & $0$   & $1$    & $9/5$  & $-1/5$ & $0$ & $17/5$ \\
  			$x_1$ & $0$ & $1$ & $0$  &  $0$    & $1/5$  & $1/5$ & $0$ & $13/5$ \\
  			$x_2$ & $0$ & $0$ & $1$ &   $0$    & $-4/5$ & $1/5$ & $0$ & $8/5$ \\
  			$x_6$ & $0$ & $0$ & $0$ &   $0$    & $4/5$ & $-1/5$ & $1$ & $-3/5$ \\
  	  \end{tabular}
  	  \end{center}
  	  The optimum solution is no longer feasible, the dual simplex algorithm is now used. 
  	  
  	  \noindent \textbf{Pivot at Row 4, Column 5:}
  	  \begin{center}
 	   \begin{tabular}{c | c | c c c c c c | c}
            		Basis & $z$ & $x_1$ & $x_2$ & $x_3$ & $x_4$ & $x_5$ & $x_6$ & RHS \\ \hline
  			           & $1$ & $0$ & $0$   & $0$ & $5$     & $0$ & $6$ & $11$  \\ \hline
  			$x_3$ & $0$ & $0$ & $0$   & $1$    & $1$  & $0$ & $-1$ & $4$ \\
  			$x_1$ & $0$ & $1$ & $0$  &  $0$    & $1$  & $0$ & $1$ & $2$ \\
  			$x_2$ & $0$ & $0$ & $1$ &   $0$    & $0$ & $0$ & $1$ & $1$ \\
  			$x_5$ & $0$ & $0$ & $0$ &   $0$    & $-4$ & $1$ & $-5$ & $3$ \\
  	  \end{tabular}
  	  \end{center}
  	   \textbf{Optimal Objective Value:} $z = 11$
  	  
  	  \noindent \textbf{Optimum Solution:} $x_1 = 2$, $x_2 = 1$
  	  
  	  \noindent This subproblem has an integer solution with an objective value greater than the current best $z$ value.
  	  The subproblem is thus fathomed and the current best $z$ value is updated.
  	  
  	  \textbf{Current Best $z$ Value:} $11$
  	  
  	  \pagebreak
  	  
  	  \noindent \section*{Subproblem 2}
  	  \textbf{Post-Optimum Tableau in Standard Form:}
  	  \begin{center}
 	   \begin{tabular}{c | c | c c c c c c | c}
            		Basis & $z$ & $x_1$ & $x_2$ & $x_3$ & $x_4$ & $x_5$ & $x_6$ & RHS \\ \hline
  			           & $1$ & $0$ & $0$   & $0$ & $1/5$     & $6/5$ & $0$ & $73/5$  \\ \hline
  			$x_3$ & $0$ & $0$ & $0$   & $1$    & $9/5$  & $-1/5$ & $0$ & $17/5$ \\
  			$x_1$ & $0$ & $1$ & $0$  &  $0$    & $1/5$  & $1/5$ & $0$ & $13/5$ \\
  			$x_2$ & $0$ & $0$ & $1$ &   $0$    & $-4/5$ & $1/5$ & $0$ & $8/5$ \\
  			$x_5$ & $0$ & $0$ & $0$ &   $0$    & $-4/5$ & $1/5$ & $1$ & $-2/5$ \\
  	  \end{tabular}
  	  \end{center}
  	  The optimum solution is no longer feasible, the dual simplex algorithm is now used. 
  	  
  	   \noindent \textbf{Pivot at Row 4, Column 4:}
  	  \begin{center}
 	   \begin{tabular}{c | c | c c c c c c | c}
            		Basis & $z$ & $x_1$ & $x_2$ & $x_3$ & $x_4$ & $x_5$ & $x_6$ & RHS \\ \hline
  			           & $1$ & $0$ & $0$   & $0$ & $0$     & $5/4$ & $1/4$ & $29/2$  \\ \hline
  			$x_3$ & $0$ & $0$ & $0$   & $1$    & $0$  & $1/4$ & $9/4$ & $5/2$ \\
  			$x_1$ & $0$ & $1$ & $0$  &  $0$    & $0$  & $1/4$ & $1/4$ & $5/2$ \\
  			$x_2$ & $0$ & $0$ & $1$ &   $0$    & $0$ & $0$ & $-1$ & $2$ \\
  			$x_4$ & $0$ & $0$ & $0$ &   $0$    & $1$ & $-1/4$ & $-5/4$ & $1/2$ \\
  	  \end{tabular}
  	  \end{center}
  	  
  	   \textbf{Optimal Objective Value:} $z = 29/2$
  	  
  	  \noindent \textbf{Optimum Solution:} $x_1 = 5/2$, $x_2 = 2$
  	  
  	  \noindent The solution is not a feasible solution to the IP and the optimal value is greater than the current best $z$ value.
  	  The subproblem is thus not fathomed and we can now branch with 
  	  $x_1 \le 2$ (Subproblem 3) and $x_1 \ge 3$ (Subproblem 4).
  	  
  	  \noindent \textbf{Current Best $z$ Value:} $11$
  	  
  	   \noindent \section*{Subproblem 3}
  	  \textbf{Post-Optimum Tableau in Standard Form:}
  	  \begin{center}
 	   \begin{tabular}{c | c | c c c c c c c | c}
            		Basis & $z$ & $x_1$ & $x_2$ & $x_3$ & $x_4$ & $x_5$ & $x_6$ & $x_7$ & RHS \\ \hline
  			           & $1$ & $0$ & $0$   & $0$ & $0$     & $5/4$ & $1/4$ & $0$ & $29/2$  \\ \hline
  			$x_3$ & $0$ & $0$ & $0$ & $1$   & $0$    & $1/4$  & $9/4$ & $0$ & $5/2$ \\
  			$x_1$ & $0$ & $1$ & $0$  &  $0$    & $0$  & $1/4$ & $1/4$ & $0$ & $5/2$ \\
  			$x_2$ & $0$ & $0$ & $1$ &   $0$    & $0$ & $0$ & $-1$ & $0$ & $2$ \\
  			$x_4$ & $0$ & $0$ & $0$ &   $0$    & $1$ & $-1/4$ & $-5/4$ & $0$ & $1/2$ \\
  			$x_7$ & $0$ & $0$ & $0$ &   $0$    & $0$ & $-1/4$ & $-1/4$ & $1$ & $-1/2$ \\
  	  \end{tabular}
  	  \end{center}
  	  The optimum solution is no longer feasible, the dual simplex algorithm is now used. 
  	  
  	  \noindent \textbf{Pivot at Row 5, Column 6} and then \textbf{Pivot at Row 1, Column 5:}
  	  \begin{center}
 	   \begin{tabular}{c | c | c c c c c c c | c}
            		Basis & $z$ & $x_1$ & $x_2$ & $x_3$ & $x_4$ & $x_5$ & $x_6$ & $x_7$ & RHS \\ \hline
  			           & $1$ & $0$ & $0$   & $1/2$ & $0$     & $0$ & $0$ & $11/2$ & $13$  \\ \hline
  			$x_5$ & $0$ & $0$ & $0$ & $-1/2$   & $0$    & $1$  & $0$ & $-9/2$ & $1$ \\
  			$x_1$ & $0$ & $1$ & $0$  &  $0$    & $0$      & $0$ & $0$ & $1$ & $2$ \\
  			$x_2$ & $0$ & $0$ & $1$ &   $1/2$    & $0$ & $0$ & $0$ & $1/2$ & $3$ \\
  			$x_4$ & $0$ & $0$ & $0$ &   $1/2$    & $1$ & $0$ & $0$ & $-1/2$ & $2$ \\
  			$x_6$ & $0$ & $0$ & $0$ &   $1/2$    & $0$ & $0$ & $1$ & $1/2$ & $1$ \\
  	  \end{tabular}
  	  \end{center}
  	  \pagebreak
  	  
  	   \textbf{Optimal Objective Value:} $z = 13$
  	  
  	  \noindent \textbf{Optimum Solution:} $x_1 = 2$, $x_2 = 3$
  	  
  	  \noindent This subproblem has an integer solution with an objective value greater than the current best $z$ value.
  	  The subproblem is thus fathomed and the current best $z$ value is updated.
  	  
  	  \noindent \textbf{Current Best $z$ Value:} $13$
  	  
  	   \noindent \section*{Subproblem 4}
  	   In this subproblem we add the constraint $x_1 \ge 3$ but this along with the constraint $x_2 \ge 2$
  	   imply that the constraint $4x_1 + x_2 \le 12$ is violated. Thus Subproblem 4 is infeasible and therefore fathomed.
  	   
  	   \noindent \section*{Conclusion}
  	   The optimum solution to the IP is $x_1 = 2, x_2 = 3$ with the objective function taking optimal value $13$.
  	   
  	   \noindent \section*{Sketch}
  	   \begin{figure}[!htb]
    		\centering
    		\includegraphics[width = \textwidth, height = 0.5\textheight]{./"images/branch and bound"}
  	  \end{figure}
  	  \noindent The preceding sketch was drawn using the now deprecated 
  	  \textit{tikz-quick-and-dirty-screenshots-with-mspaint} package.
  	  
  	  \pagebreak
  	  
  	  \item
  	  (a) Each row (including the objective row) of the optimum tableau has a righthandside value that is not an integer,
  	  and therefore a cutting plane. The cutting planes are
  	  $$\frac{1}{5}x_4 + \frac{1}{5}x_5 \ge \frac{3}{5} \mbox{ for the objective row}$$
  	  $$\frac{4}{5}x_4 + \frac{4}{5}x_5 \ge \frac{2}{5}\mbox{ for row } 1$$
  	  $$\frac{1}{5}x_4 + \frac{1}{5}x_5 \ge \frac{3}{5} \mbox{ for row } 2$$
  	  $$\frac{1}{5}x_4 + \frac{1}{5}x_5 \ge \frac{3}{5} \mbox{ for row } 3$$
  	  Thus, there are two distinct valid cutting planes:
  	  $$\frac{1}{5}x_4 + \frac{1}{5}x_5 \ge \frac{3}{5} \mbox{ and } \frac{4}{5}x_4 + \frac{4}{5}x_5 \ge \frac{2}{5}$$
  	  
  	  \noindent (b)
  	  
  	  \section*{$\mathbf{\frac{1}{5}x_4 + \frac{1}{5}x_5 \ge \frac{3}{5}}$}
  	  
  	  \textbf{Post-Optimum Tableau in Standard Form:}
  	  \begin{center}
 	   \begin{tabular}{c | c | c c c c c c | c}
            		Basis & $z$ & $x_1$ & $x_2$ & $x_3$ & $x_4$ & $x_5$ & $x_6$ & RHS \\ \hline
  			           & $1$ & $0$ & $0$   & $0$ & $1/5$     & $6/5$ & $0$ & $73/5$  \\ \hline
  			$x_3$ & $0$ & $0$ & $0$   & $1$    & $9/5$  & $-1/5$ & $0$ & $17/5$ \\
  			$x_1$ & $0$ & $1$ & $0$  &  $0$    & $1/5$  & $1/5$ & $0$ & $13/5$ \\
  			$x_2$ & $0$ & $0$ & $1$ &   $0$    & $-4/5$ & $1/5$ & $0$ & $8/5$ \\
  			$x_6$ & $0$ & $0$ & $0$ &   $0$    & $-1/5$ & $-1/5$ & $1$ & $-3/5$ \\
  	  \end{tabular}
  	  \end{center}
  	  
  	  \noindent \textbf{Pivot at Row 4, Column 4} and then \textbf{Pivot at Row 1, Column 5:}
  	   \begin{center}
 	   \begin{tabular}{c | c | c c c c c c | c}
            		Basis & $z$ & $x_1$ & $x_2$ & $x_3$ & $x_4$ & $x_5$ & $x_6$ & RHS \\ \hline
  			           & $1$ & $0$ & $0$   & $1/2$ & $0$     & $0$ & $11/2$ & $13$  \\ \hline
  			$x_5$ & $0$ & $0$ & $0$   & $-1/2$    & $0$  & $1$ & $-9/2$ & $1$ \\
  			$x_1$ & $0$ & $1$ & $0$  &  $0$    & $0$  & $0$ & $1$ & $2$ \\
  			$x_2$ & $0$ & $0$ & $1$ &   $1/2$    & $0$ & $0$ & $1/2$ & $3$ \\
  			$x_4$ & $0$ & $0$ & $0$ &   $1/2$    & $1$ & $0$ & $-1/2$ & $2$ \\
  	  \end{tabular}
  	  \end{center}
  	  
  	  \noindent \textbf{Optimum Solution:} $x_1 = 2, x_2 = 3$ with the objective function taking optimal value $13$. 
  	  
  	  \noindent \section*{$\mathbf{\frac{4}{5}x_4 + \frac{4}{5}x_5 \ge \frac{2}{5}}$}
  	  
  	  \textbf{Post-Optimum Tableau in Standard Form:}
  	  \begin{center}
 	   \begin{tabular}{c | c | c c c c c c | c}
            		Basis & $z$ & $x_1$ & $x_2$ & $x_3$ & $x_4$ & $x_5$ & $x_6$ & RHS \\ \hline
  			           & $1$ & $0$ & $0$   & $0$ & $1/5$     & $6/5$ & $0$ & $73/5$  \\ \hline
  			$x_3$ & $0$ & $0$ & $0$   & $1$    & $9/5$  & $-1/5$ & $0$ & $17/5$ \\
  			$x_1$ & $0$ & $1$ & $0$  &  $0$    & $1/5$  & $1/5$ & $0$ & $13/5$ \\
  			$x_2$ & $0$ & $0$ & $1$ &   $0$    & $-4/5$ & $1/5$ & $0$ & $8/5$ \\
  			$x_6$ & $0$ & $0$ & $0$ &   $0$    & $-4/5$ & $-4/5$ & $1$ & $-2/5$ \\
  	  \end{tabular}
  	  \end{center}
  	  
  	  \noindent \textbf{Pivot at Row 4, Column 4:}
  	   \begin{center}
 	   \begin{tabular}{c | c | c c c c c c | c}
            		Basis & $z$ & $x_1$ & $x_2$ & $x_3$ & $x_4$ & $x_5$ & $x_6$ & RHS \\ \hline
  			           & $1$ & $0$ & $0$   & $0$ & $0$     & $1$ & $1/4$ & $29/2$  \\ \hline
  			$x_3$ & $0$ & $0$ & $0$   & $1$    & $0$  & $-2$ & $9/4$ & $5/2$ \\
  			$x_1$ & $0$ & $1$ & $0$  &  $0$    & $0$  & $0$ & $1/4$ & $5/2$ \\
  			$x_2$ & $0$ & $0$ & $1$ &   $0$    & $0$ & $1$ & $-1$ & $2$ \\
  			$x_4$ & $0$ & $0$ & $0$ &   $0$    & $1$ & $1$ & $-5/4$ & $1/2$ \\
  	  \end{tabular}
  	  \end{center}
	  
  	  \noindent \textbf{Optimum Solution:} $x_1 = 5/2, x_2 = 2$ with the objective function taking optimal value $29/2$. 
  	  
  \end{enumerate}
\end{document}