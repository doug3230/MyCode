\documentclass{article}[12pt,a4paper]
%\usepackage{a4wide}
\usepackage{fullpage}
\usepackage{tikz}
\usepackage{amssymb}

\title{MA372 Submission Problems}
\author{Richard Douglas}
\date{January 24,  2014} %defaults to \today

\begin{document}
  \maketitle
  \begin{enumerate}
  	\item Consider a linear program of the form
	        $$\mbox{Maximize } z = \mathbf{c} \cdot \mathbf{x}$$
                    subject to
	        $$A\mathbf{x} \le \mathbf{b}$$
	        $$\mathbf{x} \ge 0,$$
	       where $\mathbf{b} \ge \mathbf{0}$. Prove that if $\mathbf{x}^{(1)} ,\mathbf{x}^{(2)} , \dots ,\mathbf{x}^{(k)}$ 	
	       are optimum solutions to this linear program, then any convex combination of them is also an optimum solution.
	       \newline
	       
	\item Follow the directions of the January 17 discussion problem, but use the following tableaux instead:
		(tableaux included at start of each part of my solution for Question 2.)
		\newline
	
	\item Let (LP1) denote the following program:
	        $$\mbox{Maximize } z = \mathbf{c} \cdot \mathbf{x}$$
	        subject to
	        $$A\mathbf{x} \le \mathbf{b}$$
	        $$\mathbf{x} \ge 0,$$
	         where $\mathbf{b} \ge \mathbf{0}$.
	         \begin{enumerate}
	         \item[(a)] Let (LP2) denote the linear program obtained from (LP1) by replacing $b_i$ with
		$b_i + \delta_i$, where $\delta_i \in \mathbb{R}$ for each $1 \le i \le m$. Determine conditions 
		under which an optimum basis for (LP1) is also an optimum basis for (LP2), and 	
		determine the new optimum solution in this case.
		
		\item[(b)] Let (LP3) denote the linear program obtained from (LP1) by replacing $c_j$ with
		$c_j + \Delta_j$ , where $\Delta_j \in \mathbb{R}$ for each $1 \le j \le n$
		Determine conditions under which an optimum basis for (LP1) is also an optimum basis for (LP3), 
		and determine the value of the new objective function in this case.
	         \end{enumerate}
  \end{enumerate}
  
  \begin{enumerate}
  \pagebreak
  	\item Let $k \in \mathbb{N}$ be arbitrary and suppose that $\mathbf{x}^{(1)} ,\mathbf{x}^{(2)} , \dots ,\mathbf{x}^{(k)}$
  	are optimum solutions to the linear program. Define $z^*$ to be the optimum objective value. That is to say,
  	
  	$$z^* = z(\mathbf{x}^{(1)}) =  z(\mathbf{x}^{(2)}) = \dots = z(\mathbf{x}^{(k)})$$
  	
  	Let $\mathbf x'$ be any convex combination of $\mathbf{x}^{(1)} ,\mathbf{x}^{(2)} , \dots , \mbox{ and } 
  	\mathbf{x}^{(k)}$.
  	That is, $\mathbf x'$ is of form
  	$$\mathbf x' = \Sigma_{i = 1}^{k}{t_i\mathbf{x}^{(i)}} \mbox{ where } 0 \le t_i \le 1 \mbox{ and }
  	 \Sigma_{i = 1}^{k}{t_i} = 1 $$
  
  	$\mathbf{x' \mbox{\textbf{ is feasible:}}}$
  	
  	\begin{itemize}
  	\item Clearly $\mathbf x' \ge 0$ because $\mathbf{x}^{(i)} \ge 0 \mbox{ and } t_i \ge 0 \mbox{ for all } i$
  	\item $A\mathbf x' = A(t_1\mathbf x^{(1)} + t_2\mathbf x^{(2)} + \dots + t_k\mathbf x^{(k)}) = 
  		t_1A\mathbf x^{(1)} + t_2A\mathbf x^{(2)} + \dots + t_kA\mathbf x^{(k)}$
  	\item Since $\mathbf x^{(i)}$ is feasible for all $i \mbox{ } (\Rightarrow A\mathbf x^{(i)} \le b$),  
		$$A\mathbf x' \le t_1b + t_2b + \dots + t_kb = (t_1 + t_2 + \dots + t_k)b = b$$
	\item Thus $\mathbf x'$ is feasible. \newline
  	\end{itemize}
  	
  	$\mathbf{x' \mbox{\textbf{ maximizes the objective function:}}}$
  	
  	\begin{itemize}
  	\item Since the objective function is linear,
  	$$z(\mathbf x') = z(\Sigma_{i = 1}^{k}{t_i\mathbf x^{(i)}}) = \Sigma_{i = 1}^{k}{t_iz(\mathbf x^{(i)})}
  		= \Sigma_{i = 1}^{k}{t_iz^*} = z^*\Sigma_{i = 1}^{k}{t_i} = z^*$$
	\item $z^*$ is the optimum objective value and thus $\mathbf x'$ maximizes $z$. \newline
  	\end{itemize}
  	
  	$\mathbf x'$ is feasible and maximizes $z$. Therefore, by definition, $\mathbf x'$ is an optimum solution.
  	Since $k$ and $\mathbf x'$ were arbitrary, we have that any convex combination of any finite number of
  	optimum solutions is an optimum solution.
  	\pagebreak
  	
  	\item
  	
  	(a)
		\begin{center}
 		\begin{tabular}{l | c | c c c c c c | c}
             		Basis & $z$ & $x_1$ & $x_2$ & $x_3$ & $x_4$ & $x_5$ & $x_6$ & RHS \\ \hline
  			     & $1$ & $0$ & $0$   & $-2$ & $0$  & $-3$     & $5$     & $15$  \\ \hline
  			$x_4$ & $0$ & $0$ & $0$   & $1$    & $1$      & $0$     & $-2$     & $5$ \\
  			$x_1$ & $0$ & $1$ & $0$  & $2$    & $0$      & $-1$     & $3$     & $2$ \\
  			$x_2$ & $0$ & $0$ & $1$ & $-1$    & $0$      & $-5$     & $1$     & $4$ \\
  		\end{tabular}
  		\end{center}

	This LP is unbounded because the objective row entry in column 5 is negative and all of the
	entries in the column are nonpositive. An infinite set of feasible solutions for which the objective
	function can become arbitrarily large is as follows:
	
	$$\{(x_1 = 2 + t, x_2 = 4 + 5t, x_3 = 0) \mbox{ : } t \ge 0\}$$
	
	For a given choice of $t$, this causes the objective value to become $15 + 3t$. The objective value can thus 
	be made as large as we like by choosing $t$ sufficiently large. \newline
	
	(b)
		\begin{center}
 		\begin{tabular}{l | c | c c c c c c | c}
             		Basis & $z$ & $x_1$ & $x_2$ & $x_3$ & $x_4$ & $x_5$ & $x_6$ & RHS \\ \hline
  			     & $1$ & $0$ & $0$   & $-2$ & $0$  & $-3$     & $5$     & $15$  \\ \hline
  			$x_4$ & $0$ & $0$ & $0$   & $1$    & $1$      & $0$     & $-2$     & $1$ \\
  			$x_1$ & $0$ & $1$ & $0$  & $2$    & $0$      & $1$     & $3$     & $2$ \\
  			$x_2$ & $0$ & $0$ & $1$ & $-1$    & $0$      & $2$     & $1$     & $4$ \\
  		\end{tabular}
  		\end{center}
  		
	Pivoting at row $1$, column $3$ gives us the tableau
	
		\begin{center}
 		\begin{tabular}{l | c | c c c c c c | c}
             		Basis & $z$ & $x_1$ & $x_2$ & $x_3$ & $x_4$ & $x_5$ & $x_6$ & RHS \\ \hline
  			     & $1$ & $0$ & $0$   & $0$ & $2$  & $-3$     & $1$     & $17$  \\ \hline
  			$x_3$ & $0$ & $0$ & $0$   & $1$    & $1$      & $0$     & $-2$     & $1$ \\
  			$x_1$ & $0$ & $1$ & $0$  & $0$    & $-2$      & $1$     & $7$     & $0$ \\
  			$x_2$ & $0$ & $0$ & $1$ & $0$    & $1$      & $2$     & $-1$     & $5$ \\
  		\end{tabular}
  		\end{center}
  		
	If we now pivot at row $2$, column $5$
	
		\begin{center}
 		\begin{tabular}{l | c | c c c c c c | c}
             		Basis & $z$ & $x_1$ & $x_2$ & $x_3$ & $x_4$ & $x_5$ & $x_6$ & RHS \\ \hline
  			     & $1$ & $3$ & $0$   & $0$ & $-4$  & $0$     & $22$     & $17$  \\ \hline
  			$x_3$ & $0$ & $0$ & $0$   & $1$    & $1$      & $0$     & $-2$     & $1$ \\
  			$x_5$ & $0$ & $1$ & $0$  & $0$    & $-2$      & $1$     & $7$     & $0$ \\
  			$x_2$ & $0$ & $-2$ & $1$ & $0$    & $5$      & $0$     & $-15$     & $5$ \\
  		\end{tabular}
  		\end{center}
  		
	At this point it is clear that the LP is degenerate since the bases $\{x_1, x_2, x_3\}$ and $\{x_2, x_3, x_5\}$
	both give us the same basic solution of $x_1 = 0, x_2 = 5, x_3 = 1, x_4 = 0, x_5 = 0, x_6 = 0$ \newline
	\pagebreak
	
	(c)
  		\begin{center}
 		\begin{tabular}{l | c | c c c c c c | c}
             		Basis & $z$ & $x_1$ & $x_2$ & $x_3$ & $x_4$ & $x_5$ & $x_6$ & RHS \\ \hline
  			     & $1$ & $-4$ & $0$   & $1$ & $0$  & $0$     & $2$     & $7$  \\ \hline
  			$x_4$ & $0$ & $0$ & $0$   & $1$    & $1$      & $0$     & $-2$     & $3$ \\
  			$x_5$ & $0$ & $1$ & $0$  & $2$    & $0$      & $1$     & $3$     & $2$ \\
  			$x_2$ & $0$ & $-1$ & $1$ & $-3$    & $0$      & $0$     & $-2$     & $2$ \\
  		\end{tabular}
  		\end{center}
  		
  	Pivoting at row $2$, column $1$ gives us
  		
		\begin{center}
 		\begin{tabular}{l | c | c c c c c c | c}
             		Basis & $z$ & $x_1$ & $x_2$ & $x_3$ & $x_4$ & $x_5$ & $x_6$ & RHS \\ \hline
  			     & $1$ & $0$ & $0$   & $9$ & $0$  & $4$     & $14$     & $15$  \\ \hline
  			$x_4$ & $0$ & $0$ & $0$   & $1$    & $1$      & $0$     & $-2$     & $3$ \\
  			$x_1$ & $0$ & $1$ & $0$  & $2$    & $0$      & $1$     & $3$     & $2$ \\
  			$x_2$ & $0$ & $0$ & $1$ & $-1$    & $0$      & $1$     & $1$     & $4$ \\
  		\end{tabular}
  		\end{center}
  		
	The entries of the objective row are now all nonnegative, and there is no way to create
	another optimum solution (as the objective entries in the nonbasic variable columns are positive.) 
	The unique optimum solution is thus $x_1 = 2, x_2 = 4, x_3 = 0$ where the maximum objective value is $15$. \newline
	
	(d)
  		\begin{center}
 		\begin{tabular}{l | c | c c c c c c | c}
             		Basis & $z$ & $x_1$ & $x_2$ & $x_3$ & $x_4$ & $x_5$ & $x_6$ & RHS \\ \hline
  			     & $1$ & $0$ & $0$   & $0$ & $2$  & $4$     & $0$     & $20$  \\ \hline
  			$x_3$ & $0$ & $0$ & $2$   & $1$    & $3$      & $0$     & $0$     & $5$ \\
  			$x_1$ & $0$ & $1$ & $-3$  & $0$    & $0$      & $-1$     & $0$     & $1$ \\
  			$x_6$ & $0$ & $0$ & $1$ & $0$    & $-1$      & $2$     & $1$     & $2$ \\
  		\end{tabular}
  		\end{center}
  		
	From the tableau, we see that $x_1 = 1, x_2 = 0, x_3 = 5$ is an optimum solution.
	However if we pivot at row $3$, column $2$ we get
	
		\begin{center}
 		\begin{tabular}{l | c | c c c c c c | c}
             		Basis & $z$ & $x_1$ & $x_2$ & $x_3$ & $x_4$ & $x_5$ & $x_6$ & RHS \\ \hline
  			     & $1$ & $0$ & $0$   & $0$ & $2$  & $4$     & $0$     & $20$  \\ \hline
  			$x_3$ & $0$ & $0$ & $0$   & $1$    & $5$      & $-4$     & $-2$     & $1$ \\
  			$x_1$ & $0$ & $1$ & $0$  & $0$    & $-3$      & $5$     & $3$     & $7$ \\
  			$x_2$ & $0$ & $0$ & $1$ & $0$    & $-1$      & $2$     & $1$     & $2$ \\
  		\end{tabular}
  		\end{center}
	
	And so $x_1 = 7, x_2 = 2, x_3 = 1$ is also an optimum solution. An infinite set of optimum
	solutions is thus given by the set of convex combinations
	$$\{(x_1 = t + 7(1-t), x_2 = 0t + 2(1-t), x_3 = 5t + 1(1-t)) \mbox{ : } 0 \le t \le 1\}$$
	$$= \{(x_1 = 7 - 6t, x_2 = 2(1 - t), x_3 = 4t + 1) \mbox{ : } 0 \le t \le 1\}$$
	\pagebreak
	
	\item 
	(a) Let $x_{i_1}, x_{i_2}, \dots, x_{i_n}$ be an ordered optimum basis for (LP1), and let $B$ be the
	matrix formed by the columns of (LP1)'s tableau that correspond to $x_{i_1}, x_{i_2}, \dots, x_{i_n}$ 
	with the columns of $B$ arranged in that order.
	
	Since only $b$ is changing, the operations which need to be performed in order to arrive at that basis ($B^{-1}$)
	for (LP2) remain the same. By applying the Fundamental Insight, the objective row entries all remain the same
	(meaning they all remain nonnegative), so the only thing that needs to be checked is if the basic solution is still feasible.
	
	This is true if and only if
	$$B^{-1}(b + \delta) \ge 0$$
	
	(Where $\delta$ is the column vector formed by $\delta_i$ for $i = 1, 2, \dots, n$)
	
	If this condition holds then the optimum solution for (LP2) is
	$$(x_{i_1}, x_{i_2}, \dots, x_{i_n}) = B^{-1}(b + \delta)$$ \newline
	
	(b) When $c$ is replaced by $c + \Delta$, the situation is similar to as in (a) except now
	feasibility is unaffected and what needs to be checked is that the basic solution is still optimal.
	
	Letting $c^\Delta_B = [c_{i_1} + \Delta_{i_1}, c_{i_2} + \Delta_{i_2}, \dots, c_{i_n} + \Delta_{i_n}]$
	
	Applying the Fundamental Insight gives us that the basic solution is still optimal if the objective row is
	still nonnegative which is true if and only if
	$$c^\Delta_BB^{-1}A - (c + \Delta) \ge 0 \mbox{ and } c^\Delta_BB^{-1} \ge 0$$
	
	In which case the new objective value is $c^\Delta_BB^{-1}b$
  \end{enumerate}
\end{document}