\documentclass{article}[12pt,a4paper]
%\usepackage{a4wide}
\usepackage{fullpage}
\usepackage{tikz}

\title{MA372 Submission Problems}
\author{Richard Douglas}
\date{February 7,  2014} %defaults to \today

\begin{document}
  \maketitle
  \begin{enumerate}
	\item This problem is a continuation of the Submission Problem from the January 10 Study Guide.
	
	(a) Several customers have complained that the ultra-condensed broth is too sweet, so
	the company has altered the recipe to use fewer carrots, and more onions. Specifically, 
	the recipe for ultra-condensed broth now calls for 800 grams of each of the
	three vegetables. With this new recipe, is the optimum basis for the problem still
	optimum? If so, determine the new optimum solution and objective value.
	
	(b) The company has neglected to account for the need to package the broth in its
	original description of the problem. An automated canning machine is used to
	package both the three types of vegetable broth, as well as other products produced
	by the company. The company has allocated enough time on this machine to package
	3 litres of broth per hour. Let $x_7$ denote the slack variable for this new constraint.
	Determine whether adding $x_7$ to the optimum basis for the original problem results
	in an optimum basis, and if so, determine the new optimum solution and objective
	value.
	
	\item Solve the following linear program using the Revised Simplex Algorithm, showing all steps:
	
	\begin{center} Maximize $z = 6x_1 + 10x_2 + 34x_3$ \end{center}
	subject to 
	$$2x_1 + 5x_2 + 8x_3 \le 18$$
	$$2x_1 + 4x_2 + 10x_3 \le 16$$
	$$x_1 + 3x_2 + 7x_3 \le 10$$
	$$x_i \ge 0 \mbox{ for } 1 \le i \le 3$$
	
	\item 
	(a) Let (P) be a linear program in standard form, with objective function $z$, and let
	(D) be its dual, with objective function $w$. Prove that if \textbf{x} is any feasible solution
	to (P), and \textbf{y} is any feasible solution to (D), then $z(\mathbf{x}) \le w(\mathbf{y})$. (This is called the
	Weak LP Duality Property.)
	
	(b) Consider the following linear program:
	
	\begin{center}Maximize $z = 2x_1 + x_2$\end{center}
	subject to
	$$-x_1 - 2x_2 \le −2$$
	$$-x_1 + 3x_2 \le 6$$
	$$x_1 - 4x_2 \le 4$$
	$$x_1 ,x_2 \ge 0$$
	
	Prove that there are no feasible solutions to the dual of this problem.
  \end{enumerate}
  \pagebreak
  \begin{enumerate}
  \item The starting tableau for the original problem is
  
  \begin{center}
  \begin{tabular}{l | c | c c c c c c | c}
  Basis & $z$ & $x_1$ & $x_2$ & $x_3$ & $x_4$ & $x_5$ & $x_6$ & RHS \\ \hline
           & $1$ & $-6$ & $-10$  & $-34$  & $0$    & $0$     & $0$      & $0$  \\ \hline
  	$x_4$ & $0$ & $2$ & $5$   & $8$    & $1$      & $0$     & $0$     & $18$ \\
  	$x_5$ & $0$ & $2$ & $4$  & $10$    & $0$      & $1$     & $0$     & $16$ \\
  	$x_6$ & $0$ & $1$ & $3$ & $7$    & $0$      & $0$     & $1$     & $10$ \\
  \end{tabular}
  \end{center}
  
  After the Simplex algorithm is performed, we reach the final tableau
  \begin{center}
 \begin{tabular}{l | c | c c c c c c | c}
            Basis & $z$ & $x_1$ & $x_2$ & $x_3$ & $x_4$ & $x_5$ & $x_6$ & RHS \\ \hline
  	     & $1$ & $0$ & $4$   & $0$ & $0$  & $2$     & $2$     & $52$  \\ \hline
  	$x_4$ & $0$ & $0$ & $2$   & $0$    & $1$      & $-3/2$     & $1$     & $4$ \\
  	$x_1$ & $0$ & $1$ & $-1/2$  & $0$    & $0$      & $7/4$     & $-5/2$     & $3$ \\
  	$x_3$ & $0$ & $0$ & $1/2$ & $1$    & $0$      & $-1/4$     & $1/2$     & $1$ \\
  \end{tabular}
  \end{center}
  
  With fundamental insight matrices
  $$A = \left[ \begin{tabular}{c c c}
  	$2$ & $5$ & $8$ \\
  	$2$ & $4$ & $10$ \\
  	$1$ & $3$ & $7$ 
  	\end{tabular} \right] \mbox{, }
  M = \left[ \begin{tabular}{c c c}
  	$1$ & $-3/2$ & $1$ \\
  	$0$ & $7/4$ & $-5/2$ \\
  	$0$ & $-1/4$ & $1/2$ 
  	\end{tabular} \right]$$
  	
 $$c = \left[ \begin{tabular}{c c c}
  	$6$ & $10$ & $34$
  	\end{tabular} \right]
  	\mbox{, }
 	c_B = \left[ \begin{tabular}{c c c}
  	$0$ & $6$ & $34$
  	\end{tabular} \right]
  	\mbox{, }
  	y = \left[ \begin{tabular}{c c c}
  	$0$ & $2$ & $2$
  	\end{tabular} \right]$$
  	
  (a) The change in the recipe for ultra-condensed broth changes the constraint matrix to
  $$A' = \left[ \begin{tabular}{c c c}
  	$2$ & $5$ & $8$ \\
  	$2$ & $4$ & $8$ \\
  	$1$ & $3$ & $8$ 
  	\end{tabular} \right]$$
  	
  Still using the original $M$ matrix,
  $$MA' = \left[ \begin{tabular}{c c c}
  	$0$ & $2$ & $4$ \\
  	$1$ & $-1/2$ & $-6$ \\
  	$0$ & $1/2$ & $2$ 
  	\end{tabular} \right]
  \mbox{, }
  c_B M A' - c = 
  	\left[ \begin{tabular}{c c c}
  	$0$ & $4$ & $-2$
  	\end{tabular} \right]$$
  	
  Giving us the tableau
  \begin{center}
 \begin{tabular}{l | c | c c c c c c | c}
            Basis & $z$ & $x_1$ & $x_2$ & $x_3$ & $x_4$ & $x_5$ & $x_6$ & RHS \\ \hline
  	     & $1$ & $0$ & $4$   & $-2$ & $0$  & $2$     & $2$     & $52$  \\ \hline
  	$x_4$ & $0$ & $0$ & $2$   & $4$    & $1$      & $-3/2$     & $1$     & $4$ \\
  	$x_1$ & $0$ & $1$ & $-1/2$  & $-6$    & $0$      & $7/4$     & $-5/2$     & $3$ \\
  	$x_3$ & $0$ & $0$ & $1/2$ & $2$    & $0$      & $-1/4$     & $1/2$     & $1$ \\
  \end{tabular}
  \end{center}
  
  Pivoting at row $3$, column $3$ gives us
  \begin{center}
 \begin{tabular}{l | c | c c c c c c | c}
            Basis & $z$ & $x_1$ & $x_2$ & $x_3$ & $x_4$ & $x_5$ & $x_6$ & RHS \\ \hline
  	           & $1$ & $0$ & $9/2$ & $0$   & $0$       & $7/4$     & $5/2$ & $53$  \\ \hline
  	$x_4$ & $0$ & $0$ & $1$     & $0$   & $1$      & $-1$     & $0$     & $2$ \\
  	$x_1$ & $0$ & $1$ & $1$    & $0$    & $0$      & $1$     & $-1$     & $6$ \\
  	$x_3$ & $0$ & $0$ & $1/4$ & $1$    & $0$      & $-1/8$     & $1/4$     & $1/2$ \\
  \end{tabular}
  \end{center}
  
  The RHS column and objective coefficients are still nonnegative. Thus the optimum basis is
  still optimum. The new optimum solution is $x_1 = 6$, $x_2 = 0$, $x_3 = 1/2$ with objective
  value 53.
  \pagebreak
  
  (b) The new constraint for canning is
  $$x_1 + x_2 + x_3 \le 3$$
  Converted to equality with slack variable $x_7$, this constraint becomes
  $$x_1 + x_2 + x_3 + x_7 = 3$$
  The tableau becomes
  \begin{center}
 \begin{tabular}{l | c | c c c c c c c | c}
            Basis & $z$ & $x_1$ & $x_2$ & $x_3$ & $x_4$ & $x_5$ & $x_6$ & $x_7$ & RHS \\ \hline
  	     & $1$ & $0$ & $4$   & $0$ & $0$  & $2$     & $2$ & $0$  & $52$  \\ \hline
  	$x_4$ & $0$ & $0$ & $2$ & $0$ & $1$ & $-3/2$  & $1$ & $0$ & $4$ \\
  	$x_1$ & $0$ & $1$ & $-1/2$ & $0$ & $0$ & $7/4$ & $-5/2$ & $0$ & $3$ \\
  	$x_3$ & $0$ & $0$ & $1/2$ & $1$    & $0$      & $-1/4$     & $1/2$ & $0$     & $1$ \\
  	$x_7$ & $0$ & $1$ & $1$ & $1$ & $0$ & $0$ & $0$ & $1$  & $3$
  \end{tabular}
  \end{center}
  Pivoting on columns 1 and 3 yields
   \begin{center}
 \begin{tabular}{l | c | c c c c c c c | c}
            Basis & $z$ & $x_1$ & $x_2$ & $x_3$ & $x_4$ & $x_5$ & $x_6$ & $x_7$ & RHS \\ \hline
  	     & $1$ & $0$ & $4$   & $0$ & $0$  & $2$     & $2$ & $0$  & $52$  \\ \hline
  	$x_4$ & $0$ & $0$ & $2$ & $0$ & $1$ & $-3/2$  & $1$ & $0$ & $4$ \\
  	$x_1$ & $0$ & $1$ & $-1/2$ & $0$ & $0$ & $7/4$ & $-5/2$ & $0$ & $3$ \\
  	$x_3$ & $0$ & $0$ & $1/2$ & $1$    & $0$      & $-1/4$     & $1/2$ & $0$     & $1$ \\
  	$x_7$ & $0$ & $0$ & $1$ & $0$ & $0$ & $-3/2$ & $2$ & $1$  & $-1$
  \end{tabular}
  \end{center}
  The basic solution is no longer feasible as $x_7$ is negative. 
  This means that this basis is not optimum. 
  \pagebreak
  \item
  \textbf{LP Matrices:}
  $$A = \left[ \begin{tabular}{c c c}
  	$2$ & $5$ & $8$ \\
  	$2$ & $4$ & $10$ \\
  	$1$ & $3$ & $7$ 
  	\end{tabular} \right]
  	\mbox{, }
  	b = \left[ \begin{tabular}{c}
  	$18$ \\
  	$16$ \\
  	$10$ 
  	\end{tabular} \right]
  	\mbox{, }
  	c = \left[ \begin{tabular}{c c c}
  	$6$ & $10$ & $34$  
  	\end{tabular} \right]$$
  
  \textbf{Initially:}
   $$M = \left[ \begin{tabular}{c c c}
  	$1$ & $0$ & $0$ \\
  	$0$ & $1$ & $0$ \\
  	$0$ & $0$ & $1$ 
  	\end{tabular} \right]
  	\mbox{, }
  	c_B = \left[ \begin{tabular}{c c c}
  	$0$ & $0$ & $0$
  	\end{tabular} \right]
  	\mbox{, Basis: }
  	\{x_4, x_5, x_6\}$$

  \textbf{Iteration 1:}
  $$c_BMA - c = \left[ \begin{tabular}{c c c}
  $-6$ & $-10$ & $-34$  
  \end{tabular} \right]
  \mbox{, }
  c_BM = \left[ \begin{tabular}{c c c}
  	$0$ & $0$ & $0$
  	\end{tabular} \right]
  $$
  Entering variable: $x_1$
  $$MA_1 = \left[ \begin{tabular}{c}
  	$2$ \\
  	$2$ \\
  	$1$ 
  	\end{tabular} \right]
  	\mbox{, }
  	Mb = \left[ \begin{tabular}{c}
  	$18$ \\
  	$16$ \\
  	$10$ 
  	\end{tabular} \right]
  	\mbox{, }$$
  Leaving variable: $x_5$
  $$
   \left[ \begin{tabular}{c | c c c}
  	$2$ & $1$ & $0$ & $0$ \\
  	$2$ & $0$ & $1$ & $0$ \\
  	$1$ & $0$ & $0$ & $1$ 
  	\end{tabular} \right]
  	\rightarrow
   \left[ \begin{tabular}{c | c c c}
  	$0$ & $1$ & $-1$ & $0$ \\
  	$1$ & $0$ & $1/2$ & $0$ \\
  	$0$ & $0$ & $-1/2$ & $1$ 
  	\end{tabular} \right]
  $$
  Result:
   $$M = \left[ \begin{tabular}{c c c}
  	$1$ & $-1$ & $0$ \\
  	$0$ & $1/2$ & $0$ \\
  	$0$ & $-1/2$ & $1$ 
  	\end{tabular} \right]
  	\mbox{, }
  	c_B = \left[ \begin{tabular}{c c c}
  	$0$ & $6$ & $0$
  	\end{tabular} \right]
  	\mbox{, Basis: }
  	\{x_4, x_1, x_6\}$$
  
  \textbf{Iteration 2:}
   $$c_BMA - c = \left[ \begin{tabular}{c c c}
  $0$ & $2$ & $-4$  
  \end{tabular} \right]
  \mbox{, }
  c_BM = \left[ \begin{tabular}{c c c}
  	$0$ & $3$ & $0$
  	\end{tabular} \right]
  $$
   Entering variable: $x_3$
  $$MA_3 = \left[ \begin{tabular}{c}
  	$-2$ \\
  	$5$ \\
  	$2$ 
  	\end{tabular} \right]
  	\mbox{, }
  	Mb = \left[ \begin{tabular}{c}
  	$2$ \\
  	$8$ \\
  	$2$ 
  	\end{tabular} \right]
  	\mbox{, }$$
  Leaving variable: $x_6$
  $$
   \left[ \begin{tabular}{c | c c c}
  	$-2$ & $1$ & $-1$ & $0$ \\
  	$5$ & $0$ & $1/2$ & $0$ \\
  	$2$ & $0$ & $-1/2$ & $1$ 
  	\end{tabular} \right]
  	\rightarrow
   \left[ \begin{tabular}{c | c c c}
  	$0$ & $1$ & $-3/2$ & $1$ \\
  	$0$ & $0$ & $7/4$ & $-5/2$ \\
  	$1$ & $0$ & $-1/4$ & $1/2$ 
  	\end{tabular} \right]
  $$
  Result:
   $$M = \left[ \begin{tabular}{c c c}
  	$1$ & $-3/2$ & $1$ \\
  	$0$ & $7/4$ & $-5/2$ \\
  	$0$ & $-1/4$ & $1/2$ 
  	\end{tabular} \right]
  	\mbox{, }
  	c_B = \left[ \begin{tabular}{c c c}
  	$0$ & $6$ & $34$
  	\end{tabular} \right]
  	\mbox{, Basis: }
  	\{x_4, x_1, x_3\}$$
  
  \textbf{Iteration 3:}
   $$c_BMA - c = \left[ \begin{tabular}{c c c}
  $0$ & $4$ & $0$  
  \end{tabular} \right]
  \mbox{, }
  c_BM = \left[ \begin{tabular}{c c c}
  	$0$ & $2$ & $2$
  	\end{tabular} \right]
  $$
  Algorithm terminated. \newline
  
  \textbf{Optimum solution:}
  $$Mb = \left[ \begin{tabular}{c}
  	$4$ \\
  	$3$ \\
  	$1$ 
  	\end{tabular} \right]
  	\mbox{, }
  	x_1 = 3, x_2 =0, x_3 = 1$$
  \textbf{Objective value:}
  $$c_BMb = 52$$
  \pagebreak
  \item
  (a) Suppose that \textbf{x} is a feasible solution for (P).  This implies $A\mathbf{x} \le b$ and $\mathbf{x} \ge 0$.
  
  Suppose also that \textbf{y} is a feasible solution for (D). This implies $\mathbf{y}A \ge c$ and $\mathbf{y} \ge 0$.
  
  Therefore
  $$w(\mathbf{y}) = \mathbf{y}b \ge \mathbf{y}(A\mathbf{x}) = (\mathbf{y}A)\mathbf{x} \ge c\mathbf{x} = z(\mathbf{x})$$
  
  Thus a feasible solution for (D) can be used to obtain an upper bound on the objective value for (P). \newline
  
  (b) The dual problem to this LP is
  
  \begin{center} Minimize $w = -2y_1 + 6y_2 + 4y_3$\end{center}
  subject to
  $$-y_1 - y_2 + y_3 \ge 2$$
  $$-2y_1 + 3y_2 - 4y_3 \ge 1$$
  $$y_1, y_2, y_3 \ge 0$$
  If the first two inequalities hold, then we can add 3 times the first to the second to get
  $$-5y_1 - y_3 \ge 7$$
  but this is impossible since $y_1$ and $y_3$ are both nonnegative. 
  
  Thus the structural constraints imply that the nonnegativity constraints do not hold.
  
  Thus there are no feasible solutions to the dual problem.
  
  \end{enumerate}
\end{document}