\documentclass{article}[12pt,a4paper]
%\usepackage{a4wide}
\usepackage{fullpage}
\usepackage{tikz}

\title{MA372 Submission Problems}
\author{Richard Douglas}
\date{February 14,  2014}

\begin{document}
  \maketitle
  \begin{enumerate}
  \item For an LP of the form maximize $z = cx$ subject to $Ax \le b$, $x \ge 0$, with $n$ decision variables
  and $m$ constraints, prove the \textit{Complementary Slackness Theorem}: let $x$ be a feasible solution
  to the LP, and let $y$ be a feasible solution to its dual. Then $x$ and $y$ are optimum solutions to their 
  respective problems if and only if
  $$x_j(\sum_{1 \le i \le m}{a_{ij}y_i - c_j}) = 0$$
  for all $1 \le j \le m$, and
  $$y_i(\sum_{1 \le j \le n}{a_{ij}x_j - b_i}) = 0$$ 
  for all $1 \le i \le n$.
  \newline
  \item 
  (a) In part (a) of the Submission Problem in the January 22 Study Guide, you saw that the change in resources
  resulted in a tableau with an infeasible basic solution:
  \begin{center}
  \begin{tabular}{c | c | c c c c c c | c}
  Basis & $z$ & $x_1$ & $x_2$ & $x_3$ & $x_4$ & $x_5$ & $x_6$ & $RHS$ \\ \hline
           & $1$ & $0$ & $4$ & $0$ & $0$ & $2$ & $2$ & $72$ \\ \hline
  $x_4$ & $0$ & $0$ & $2$ & $0$ & $1$ & $-3/2$ & $1$ & $4$ \\
  $x_1$ & $0$ & $1$ & $-1/2$ & $0$ & $0$ & $7/4$ & $-5/2$ & $-5$ \\
  $x_3$ & $0$ & $0$ & $1/2$ & $1$ & $0$ & $-1/4$ & $1/2$ & $3$
  \end{tabular}
  \end{center}
   Use the Dual Simplex Algorithm to return to a tableau with a feasible solution, and find the new optimum solution. \newline
  
  (b) Apply the Dual Simplex Algorithm to solve the following Linear Program:
  $$\mbox{Minimize } 6x_1 + x_2 + 2x_3$$
  subject to
  $$3x_1 + x_2 + x_3 \ge 2$$
  $$x_1 - x_2 + x_3 \ge -1$$
  $$x_1 + 2x_2 - x_3 \ge 1$$
  $$x_1, x_2, x_3 \ge 0$$
  \pagebreak
  
  \item
  (a) Solve the following LP using the ``Big M'' method.
  $$\mbox{Minimize } 40x_1 + 30x_2 + 10x_3$$
  subject to
  $$x_1 + x_2 + x_3 = 6$$
  $$x_1 + 2x_2 + 4x_3 \le 12$$
  $$x_1, x_2, x_3 \ge 0$$ \newline
  (b) Use the ``Big M'' method to demonstrate that the following LP has no feasible solutions:
  $$\mbox{Maximize } z = 5x_1 + 4x_2$$
  subject to
  $$3x_1 + 2x_2 \le 6$$
  $$2x_1 - x_2 \ge 6$$
  $$x_1, x_2 \ge 0$$ \newline
  \end{enumerate}
  \textbf{Note:} In my tableaux I put a $-1$ in the $z$ entry of the objective row whenever I am maximizing
  $-z$. I do this so that I remember to multiply the objective value for $-z$ by $-1$ at the end of the algorithm.
  I also denote artificial variables as $a_1$, $a_2$, $a_3$, etc.... This is to remind me that we don't want such
  variables to be in the optimum basis. I will change notation if it would be preferred.
  \pagebreak
  
  \begin{enumerate}
  \item \textbf{Proof:} \newline
  ``$\mathbf{\Leftarrow}$'' \newline
  The sums can be rewritten as
  $$\sum_{1 \le i \le m}{(x_ja_{ij}y_i - c_jx_j)} = 0$$
  for all $1 \le j \le m$, and
  $$\sum_{1 \le j \le n}{(x_ja_{ij}y_i - b_iy_i)} = 0$$ 
  for all $1 \le i \le n$.
  
  It follows then that if we sum over all $j$ for the first sum equations and all $i$ for the second sum equations 
  then the righthandside will still be $0$.
  $$\sum_{1 \le j \le n}{\sum_{1 \le i \le m}{(x_ja_{ij}y_i - c_jx_j})} = 0$$
  $$\sum_{1 \le i \le m}{\sum_{1 \le j \le n}{(x_ja_{ij}y_i - b_iy_i)}} = 0$$ 
  Simplifying gives us
  $$\sum_{1 \le j \le n}{\sum_{1 \le i \le m}{x_ja_{ij}y_i}} - \sum_{1 \le j \le n}{c_jx_j} = 0$$
  $$\sum_{1 \le i \le m}{\sum_{1 \le j \le n}{x_ja_{ij}y_i - \sum_{1 \le i \le m}{b_iy_i}}} = 0$$ 
  The order of summation in the double sum does not change its value. Thus
  $$\sum_{1 \le j \le n}{c_jx_j} = \sum_{1 \le i \le m}{b_iy_i}$$
  By the Fundamental Theorem of Optimization, $x$ and $y$ are solutions to their respective problems. \newline
  
  ``$\mathbf{\Rightarrow}$'' \newline
  Since $x$ and $y$ are optimal, we can use the fact that
  $$\sum_{1 \le j \le n}{c_jx_j} = \sum_{1 \le i \le m}{b_iy_i}$$
  It follows that
  $$ -\sum_{1 \le j \le n}{\sum_{1 \le i \le m}{x_ja_{ij}y_i}} + \sum_{1 \le j \le n}{c_jx_j} +
  \sum_{1 \le i \le m}{\sum_{1 \le j \le n}{x_ja_{ij}y_i - \sum_{1 \le i \le m}{b_iy_i}}} = 0$$
  Factoring gives us
  $$\sum_{1 \le j \le n}{x_j(\sum_{1 \le i \le m}{c_j - y_ia_{ij}})} + 
      \sum_{1 \le i \le m}{y_i(\sum_{1 \le j \le n}{a_{ij}x_j - b_i})} = 0$$
  \pagebreak
  
  $x$ and $y$ are feasible in their respective problems. Therefore
  $$Ax \le b \Rightarrow \sum_{1 \le j \le n}{a_{ij}x_j} \le b_j \mbox{ for all } j \Rightarrow
   b_j - \sum_{1 \le j \le n}{a_{ij}x_j} \ge 0 \mbox{ for all } j \le 0$$ 
  $$yA \ge c \Rightarrow \sum_{1 \le i \le m}{y_ia_{ij}} \ge c_i \mbox{ for all } i \Rightarrow
   \sum_{1 \le i \le m}{y_ia_{ij}} - c_i \ge 0 \mbox{ for all } i$$   
  Now in the previous equation
  $$\sum_{1 \le j \le n}{x_j(\sum_{1 \le i \le m}{c_j - y_ia_{ij}})} + 
      \sum_{1 \le i \le m}{y_i(\sum_{1 \le j \le n}{a_{ij}x_j - b_i})} = 0$$
  $x_j$, $y_i$, $\sum_{1 \le i \le m}{(y_ia_{ij} - c_j)}$, and $\sum_{1 \le j \le n}{(a_{ij}x_j - b_i)}$ are all 
  nonnegative. Therefore it must be the case that each of the individual summations holds.
  
  That is we must have
  $$x_j(\sum_{1 \le i \le m}{c_j - a_{ij}y_i}) = x_j(\sum_{1 \le i \le m}{a_{ij}y_i - c_j}) = 0$$
  for all $1 \le j \le m$, and
  $$y_i(\sum_{1 \le j \le n}{a_{ij}x_j - b_i}) = 0$$ 
  for all $1 \le i \le n$. \newline
  
  With ``$\mathbf{\Leftarrow}$' and  ``$\mathbf{\Rightarrow}$'' proven, the result follows.
  
  \pagebreak
  
  \item
  (a) Starting with 
   \begin{center}
  \begin{tabular}{c | c | c c c c c c | c}
  Basis & $z$ & $x_1$ & $x_2$ & $x_3$ & $x_4$ & $x_5$ & $x_6$ & $RHS$ \\ \hline
           & $1$ & $0$ & $4$ & $0$ & $0$ & $2$ & $2$ & $72$ \\ \hline
  $x_4$ & $0$ & $0$ & $2$ & $0$ & $1$ & $-3/2$ & $1$ & $4$ \\
  $x_1$ & $0$ & $1$ & $-1/2$ & $0$ & $0$ & $7/4$ & $-5/2$ & $-5$ \\
  $x_3$ & $0$ & $0$ & $1/2$ & $1$ & $0$ & $-1/4$ & $1/2$ & $3$
  \end{tabular}
  \end{center}
  For the Dual Simplex Algorithm, we wish to pivot on row $2$ due to the negative entry in the $b$ column. 
  The Dual Minimum Ratio Test says to pivot at column $6$ since $|\frac{2}{-5/2}| < |\frac{4}{-1/2}|$. \newline 
  The resulting tableau is
  \begin{center}
  \begin{tabular}{c | c | c c c c c c | c}
  Basis & $z$ & $x_1$ & $x_2$ & $x_3$ & $x_4$ & $x_5$ & $x_6$ & $RHS$ \\ \hline
           & $1$ & $4/5$ & $18/5$ & $0$ & $0$ & $17/5$ & $0$ & $68$ \\ \hline
  $x_4$ & $0$ & $2/5$ & $9/5$ & $0$ & $1$ & $-4/5$ & $0$ & $2$ \\
  $x_6$ & $0$ & $-2/5$ & $1/5$ & $0$ & $0$ & $-7/10$ & $1$ & $2$ \\
  $x_3$ & $0$ & $1/5$ & $2/5$ & $1$ & $0$ & $1/10$ & $0$ & $2$
  \end{tabular}
  \end{center}
  Thus the new optimum solution is \{$x_1 = 0$, $x_2 = 0$, $x_3 = 2$\} with objective value $68$. \newline
  
  (b)
  The linear program is first converted to the maximization problem
   $$\mbox{Maximize } -6x_1 - x_2 - 2x_3$$
  subject to
  $$-3x_1 - x_2 - x_3 + x_4 = -2$$
  $$-x_1 + x_2 - x_3 + x_5 = 1$$
  $$-x_1 - 2x_2 + x_3 + x_6 = -1$$
  $$x_1, x_2, x_3, x_4, x_5, x_6 \ge 0$$
  where $x_4$, $x_5$, $x_6$ are slack variables. \newline
  
  The starting tableau is
  \begin{center}
  \begin{tabular}{c | c | c c c c c c | c}
  Basis & $z$ & $x_1$ & $x_2$ & $x_3$ & $x_4$ & $x_5$ & $x_6$ & $RHS$ \\ \hline
           & $-1$ & $6$ & $1$ & $2$ & $0$ & $0$ & $0$ & $0$ \\ \hline
  $x_4$ & $0$ & $-3$ & $-1$ & $-1$ & $1$ & $0$ & $0$ & $-2$ \\
  $x_5$ & $0$ & $-1$ & $1$ & $-1$ & $0$ & $1$ & $0$ & $1$ \\
  $x_6$ & $0$ & $-1$ & $-2$ & $1$ & $0$ & $0$ & $1$ & $-1$
  \end{tabular}
  \end{center}
  The basic solution is not feasible because of the negative entry in the final column of row 3.
  If we pivot on row 3, the Dual Minimum Ratio Test says to only consider columns with a negative
  entry in that row. Thus column 3 is not considered. $|\frac{1}{-2}| < |\frac{6}{-1}|$ therefore
  column 2 is selected. 
  
   Pivot at row $3$, column $2$
   \begin{center}
  \begin{tabular}{c | c | c c c c c c | c}
  Basis & $z$ & $x_1$ & $x_2$ & $x_3$ & $x_4$ & $x_5$ & $x_6$ & $RHS$ \\ \hline
           & $-1$ & $11/2$ & $0$ & $5/2$ & $0$ & $0$ & $1/2$ & $-1/2$ \\ \hline
  $x_4$ & $0$ & $-5/2$ & $0$ & $-3/2$ & $1$ & $0$ & $-1/2$ & $-3/2$ \\
  $x_5$ & $0$ & $-3/2$ & $0$ & $-1/2$ & $0$ & $1$ & $1/2$ & $1/2$ \\
  $x_2$ & $0$ & $1/2$ & $1$ & $-1/2$ & $0$ & $0$ & $-1/2$ & $1/2$
  \end{tabular}
  \end{center}  
  
  The basic solution still is not feasible because of the negative entry in the final column of row 1.
  For pivoting on row 1, the Dual Minimum Ratio Test says to choose column 6 since 
  $|\frac{1/2}{-1/2}| < |\frac{5/2}{-3/2}|$ and $|\frac{1/2}{-1/2}| < |\frac{11/2}{-5/2}|$.
  \pagebreak
  
  Pivot at row $1$, column $6$
   \begin{center}
  \begin{tabular}{c | c | c c c c c c | c}
  Basis & $z$ & $x_1$ & $x_2$ & $x_3$ & $x_4$ & $x_5$ & $x_6$ & $RHS$ \\ \hline
           & $-1$ & $3$ & $0$ & $1$ & $1$ & $0$ & $0$ & $-2$ \\ \hline
  $x_6$ & $0$ & $5$ & $0$ & $3$ & $-2$ & $0$ & $1$ & $3$ \\
  $x_5$ & $0$ & $-4$ & $0$ & $-2$ & $1$ & $1$ & $0$ & $-1$ \\
  $x_2$ & $0$ & $3$ & $1$ & $1$ & $-1$ & $0$ & $0$ & $2$
  \end{tabular}
  \end{center}  
  Applying similar reasoning,
  
  Pivot at row $2$, column $3$
  \begin{center}
  \begin{tabular}{c | c | c c c c c c | c}
  Basis & $z$ & $x_1$ & $x_2$ & $x_3$ & $x_4$ & $x_5$ & $x_6$ & $RHS$ \\ \hline
           & $-1$ & $1$ & $0$ & $0$ & $3/2$ & $1/2$ & $0$ & $-5/2$ \\ \hline
  $x_6$ & $0$ & $-1$ & $0$ & $0$ & $-1/2$ & $3/2$ & $1$ & $3/2$ \\
  $x_3$ & $0$ & $2$ & $0$ & $1$ & $-1/2$ & $-1/2$ & $0$ & $1/2$ \\
  $x_2$ & $0$ & $1$ & $1$ & $0$ & $-1/2$ & $1/2$ & $0$ & $3/2$
  \end{tabular}
  \end{center}  
  All entries in the final column are nonnegative. This tells us that the basic solution is now feasible
  and thus also optimum (as the objective row entries are also nonnegative.) The Dual Simplex Algorithm 
  terminates.
  
  The optimum solution is \{$x_1 = 0$, $x_2 = 3/2$, $x_3 = 1/2$\} with objective value $5/2$.
   
  
  \pagebreak
  \item
  (a) The LP can be viewed as the maximization problem:
  $$\mbox{Maximize } -z = -40x_1 - 30x_2 - 10x_3 - 1000a_1$$
  subject to
  $$x_1 + 2x_2 + 4x_3 + x_4 = 12$$
  $$x_1 + x_2 + x_3 + a_1 = 6$$
  $$x_1, x_2, x_3, x_4, a_1 \ge 0$$
  where $x_4$ is a slack variable and $a_1$ is an artificial variable introduced so that we may have a valid starting basis for the 
  simplex algorithm. The current tableau is thus
   \begin{center}
  \begin{tabular}{c | c | c c c c c | c}
  Basis & $z$ & $x_1$ & $x_2$ & $x_3$ & $x_4$ & $a_1$ & $RHS$ \\ \hline
           & $-1$ & $40$ & $30$ & $10$ & $0$ & $1000$ & $0$ \\ \hline
  $x_4$ & $0$ & $1$ & $2$ & $4$ & $1$ & $0$ & $12$ \\
  $a_1$ & $0$ & $1$ & $1$ & $1$ & $0$ & $1$ & $6$
  \end{tabular}
  \end{center}
  The basis $\{x_4, a_1\}$ is not valid until after we perform Gaussian elimination to remove the $1000$ from the objective row.
  The starting tableau is thus
  \begin{center}
  \begin{tabular}{c | c | c c c c c | c}
  Basis & $z$ & $x_1$ & $x_2$ & $x_3$ & $x_4$ & $a_1$ & $RHS$ \\ \hline
           & $-1$ & $-960$ & $-970$ & $-990$ & $0$ & $0$ & $-6000$ \\ \hline
  $x_4$ & $0$ & $1$ & $2$ & $4$ & $1$ & $0$ & $12$ \\
  $a_1$ & $0$ & $1$ & $1$ & $1$ & $0$ & $1$ & $6$
  \end{tabular}
  \end{center}
  Pivot at row $2$, column $1$
   \begin{center}
  \begin{tabular}{c | c | c c c c c | c}
  Basis & $z$ & $x_1$ & $x_2$ & $x_3$ & $x_4$ & $a_1$ & $RHS$ \\ \hline
           & $-1$ & $0$ & $-10$ & $-30$ & $0$ & $960$ & $-240$ \\ \hline
  $x_4$ & $0$ & $0$ & $1$ & $3$ & $1$ & $-1$ & $6$ \\
  $x_1$ & $0$ & $1$ & $1$ & $1$ & $0$ & $1$ & $6$
  \end{tabular}
  \end{center}
  Pivot at row $1$, column $3$
   \begin{center}
  \begin{tabular}{c | c | c c c c c | c}
  Basis & $z$ & $x_1$ & $x_2$ & $x_3$ & $x_4$ & $a_1$ & $RHS$ \\ \hline
           & $-1$ & $0$ & $0$ & $0$ & $10$ & $950$ & $-180$ \\ \hline
  $x_3$ & $0$ & $0$ & $1/3$ & $1$ & $1/3$ & $-1/3$ & $2$ \\
  $x_1$ & $0$ & $1$ & $2/3$ & $0$ & $-1/3$ & $4/3$ & $4$
  \end{tabular}
  \end{center}
  This solution is optimal. However if we pivot at row $1$, column $2$, we get
  \begin{center}
  \begin{tabular}{c | c | c c c c c | c}
  Basis & $z$ & $x_1$ & $x_2$ & $x_3$ & $x_4$ & $a_1$ & $RHS$ \\ \hline
           & $-1$ & $0$ & $0$ & $0$ & $10$ & $950$ & $-180$ \\ \hline
  $x_2$ & $0$ & $0$ & $1$ & $3$ & $1$ & $-1$ & $6$ \\
  $x_1$ & $0$ & $1$ & $0$ & $-2$ & $-1$ & $2$ & $0$
  \end{tabular}
  \end{center}
  The LP is degenerate with two possible solutions being \{$x_1 = 4$, $x_2 = 0$, $x_3 = 2$\} and \linebreak
  \{$x_1 = 0$, $x_2 = 6$, $x_3 = 0$\}. In any case, the objective value is $180$.
  \pagebreak
  \item[3.] (b) The problem can be viewed as
  $$\mbox{Maximize } z = 5x_1 + 4x_2 - Ma_1$$
  subject to
  $$3x_1 + 2x_2 + x_3 = 6$$
  $$2x_1 - x_2 - x_4 + a_1 = 6$$
  $$x_1, x_2, x_3, x_4, a_1 \ge 0$$
  where $x_3$ is a slack variable, $x_4$ is a surplus variable, $a_1$ is an artificial variable, and 
  $M$ is a large positive constant. This gives us the tableau
  \begin{center}
  \begin{tabular}{c | c | c c c c c | c}
  Basis & $z$ & $x_1$ & $x_2$ & $x_3$ & $x_4$ & $a_1$ & $RHS$ \\ \hline
           & $1$ & $-5$ & $-4$ & $0$ & $0$ & $M$ & $0$ \\ \hline
  $x_3$ & $0$ & $3$ & $2$ & $1$ & $0$ & $0$ & $6$ \\
  $a_1$ & $0$ & $2$ & $-1$ & $0$ & $-1$ & $1$ & $6$
  \end{tabular}
  \end{center}
  Performing Gaussian elimination to get a valid basis gives us the starting tableau
  \begin{center}
  \begin{tabular}{c | c | c c c c c | c}
  Basis & $z$ & $x_1$ & $x_2$ & $x_3$ & $x_4$ & $a_1$ & $RHS$ \\ \hline
           & $1$ & $-5-2M$ & $M-4$ & $0$ & $M$ & $0$ & $-6M$ \\ \hline
  $x_3$ & $0$ & $3$ & $2$ & $1$ & $0$ & $0$ & $6$ \\
  $a_1$ & $0$ & $2$ & $-1$ & $0$ & $-1$ & $1$ & $6$
  \end{tabular}
  \end{center}
  Since $M$ is a large positive constant, only the objective entry in column $1$ is negative.
  Thus the next step is to pivot at row $1$, column $1$ which gives us
   \begin{center}
  \begin{tabular}{c | c | c c c c c | c}
  Basis & $z$ & $x_1$ & $x_2$ & $x_3$ & $x_4$ & $a_1$ & $RHS$ \\ \hline
           & $1$ & $0$ & $(7M - 2)/3$ & $(5 + 2M)/3$ & $M$ & $0$ & $10-2M$ \\ \hline
  $x_1$ & $0$ & $1$ & $2/3$ & $1/3$ & $0$ & $0$ & $2$ \\
  $a_1$ & $0$ & $0$ & $-7/3$ & $-2/3$ & $-1$ & $1$ & $2$
  \end{tabular}
  \end{center}
  At this point the simplex algorithm would terminate since a large value for $M$ would cause all entries in
  the objective row (with the exception of the objective value) to be nonnegative. 
  However the artificial variable $a_1$ is part of the basis if $M$ is sufficiently large. 
  This implies that there are no feasible solutions to the LP.
  
  \end{enumerate}
\end{document}